% !TEX root = nov.tex
\section{Preparation theorem} \label{sec:prep}

In this section we work in a relative $P$-minimal structure $(K,\Gamma_K)$ where $K$ is a $p$-adically closed field, i.e., a valued field which is elementarily equivalent to a finite extension of $\mathbb{Q}_p$. Although the applications in Sections 3 and 4 are restricted to $p$-adic fields, the results in this section hold for general $p$-adically closed fields. The following is a version of the well-known cell decomposition for semi-algebraic sets. 

\begin{prop}[Semi-algebraic preparation theorem, Theorem 3.3.2 in \cite{Clu-Gor-Hal-14}]\label{prop.celldecomp}
Let $X \subseteq  S \times K$ and $f_j:X \to\Gamma_K $ be $L_{ring,2}$-definable functions for $j=1,\ldots,r$. Then there exists a finite partition of $X$ into $p$-adic cells $C_i$ 
\[C_i = \left\{(s,t) \in D\times K \ \left| \ \begin{array}{l} \alpha_i(s) \square_1 \ \ord(t-c_i(s)) \ \square_2 \ \beta_i(s),\\ \ord(t-c_i(s)) \equiv k_i\mod n_i,\\ \ac_{m_i}(t-c_i(s)) = \xi_i \end{array} \right\}\right.,\]
such that for each occurring cell $C_i$ one has that for each $(s,t)\in C_i$
\[f_j(s,t)=  h_{ij}(s)+ a_{ij}\frac{\ord(t- c_i(s))- k_i}{n_i},
\]
for $a_{ij}$ integers and $h_{ij}: D\to \Gamma_K$ $L_{ring,2}$-definable functions for $j=1,\ldots,r$. 
\end{prop}

An analogous statement is true where semi-algebraic is replaced by sub-analytic.  

\begin{lem}\label{lem.semialgfnc}
Let $f: X \subseteq K \to \Gamma_K$ be an definable function. Then, there exists an $\cL_{Ring,2}$-definable function {$h: X\to K$} such that \[f(x) = \ord h(x)\] for all $x \in X$. 
\end{lem}
\begin{proof}

It follows from our assumptions that $\mathrm{Graph}(f)$ is an $\Lm_{\text{ring}, 2}$-definable subset of $K \times \Gamma_K$. Now consider the related set
\[ G(f) := \{(x,y) \in K^2 \mid (x, \ord y) \in \mathrm{Graph}(f)\}.\]
Clearly, G(f) is semi-algebraic, that is, $\cL_{ring}$-definable. Since the structure of semi-algebraic sets admits definable Skolem functions by a result of van den Dries \cite{vdd-84}, one can find an $\Lring$-definable function $h: \Pi_x(G(f)) \to K$, such that $(x, h(x)) \in G(f)$. 
\end{proof}


\begin{thm}[K-Preparation theorem]\label{thm:partialprep} 
Let $(K,\Gamma_K)$ be relatively $P$-minimal $\Lm_2$-structure. Let $S\subseteq K^l\times\Gamma^l$ and $X \subseteq S \times K$ be $\Lm_2$-definable sets, and $f:X\rightarrow \Gamma_k$ be an $\Lm_2$-definable function. There exists a $K$-cell decomposition $\{(\Sigma_i)_{i}, (C_{\delta_{ij}})_{i,j})\}$ of $X$, such that for each part $X_i$ in the decomposition and $S_i=\pi_S(X_i)$, there is a tuple of $\Lm_2$-definable functions $\gamma_{ij}:S_i\to\Gamma_K$ and integers $a_{ij}$ satisfying that: for all $s\in S_i$, and for any choice of  $(c_{1,s}, \ldots, c_{r_i,s}) \in (\Sigma_i)_s$, we have that for every $t \in C_{\delta_{ij}}(t,c_j,\alpha_{ij},\beta_{ij};s)$, 
\[f(s,t) = \gamma_{ij}(s) + a_{ij}\frac{\ord(t-c_{j,s})-k_{ij}}{n},\]
where $\delta_{ij} =(\Box, k_{ij},n,m,\xi)$.
%We stress that the functions $c: S\to K$ are NOT necessarily $\Lm_2$-definable {\color{red} EXPLAIN WHAT THIS MEANS AND THAT IT IS BETTER THAN IT SOUNDS}.
\end{thm}


\begin{proof}[Proof of Theorem \ref{thm:partialprep}]

Fix a parameter $s \in \Pi_S(X)$. By Lemma \ref{lem.semialgfnc} there is a $\Lring$-definable function {$h_s: X_s\to K$} such that $f_s(t) = \ord h_s(t)$ for all $t \in X_s$ (we do not assume that $h_s$ is uniformly definable for all $s\in \Pi_S(X)$). By Proposition \ref{prop.celldecomp}, there exists a finite partition of $X_s$ into cells  
\begin{equation*}
C_s = \left\{t \in K \ \left| \ \begin{array}{l} \alpha_s \square_1 \ \ord(t-c(s)) \ \square_2 \ \beta_s,\\ \ord(t-c_s) \equiv k_s\mod n_s,\\ \ac_{m_s}(t-c_s) = \xi_{m_s,s} \end{array} \right\}\right.,
\end{equation*}
where $\alpha_s, \beta_s \in \Gamma_k, c_s \in K$, $(\square,k_s, n_s, m_s,\xi_{m_s,s})\in P_K$ and on each cell $C_s$ 
\begin{equation}\label{eq:preparation}
f_s(t)=\ord h_s(t) =\gamma_s + a_s\frac{\ord(t-c_s)-k_s}{n_s},
\end{equation}
for some $a_s\in \ZZ$ and $\gamma_s\in \Gamma_K$. Notice that in the cell decomposition of $X_s$ there might appear cells of the same type. 

\begin{claim}\label{claim:compact} 
There is a natural number $N\geq 1$ such that for every $s\in S$, there is a cell decomposition of $X_s$ into less than $N$-cells, such that each cell satisfies \ref{eq:preparation} and $n_s,m_s,|a_s|<N$.
\end{claim}

\begin{proof}[Proof of the claim:]

The claim will follow by a standard compactness argument. For a positive integer $N$, let 
\begin{align}\label{defPK}
\begin{split}
P_{K,N}&:= \{(\square, k,n,m,\xi)\in P_K \mid n<N, m<N\}\\
E_{K,N} &:= \bigsqcup_{i=1}^N P_{K,N},
\end{split}
\end{align}
the disjoint union of $N$ copies of $P_{K,N}$. Notice that $E_{K,N}$ is a finite set. For $\delta=(\square, k,n,m,\xi)\in P_{K,N}$ we let $k_{\delta}:=k$, $n_\delta:=n$, and so forth. Given $J\subseteq E_{K,N}$, $\{\delta_1,\ldots,\delta_{|J|}\}$ a fixed enumeration of $J$, $y\in K^{|J|}$, $\alpha=(\alpha_1,\alpha_2)\in \Gamma_K^{2|J|}$ and using the formulas defined in \ref{cellformula}, we put 
\[C_J(y,\alpha)=\bigcup_{i=1}^{|J|} C_{\delta_i}(K,y_i,\alpha_{1i},\alpha_{2i}).
\]
Let $a\in\ZZ^{|J|}$ be such that such that $|a_i|<N$ for all $1\leq i\leq |J|$; let $t,x,y$ be variables of type $K$, $t$ of length 1, $x$ of the same length as elements in $S$ and $y$ of length $|J|$; let $\alpha=(\alpha_1,\alpha_2,\alpha_3)$ be variables of type $\Gamma_K$, all $\alpha_1,\alpha_2$ and $\alpha_3$ of length $|J|$. We define the formula $\phi_{J,a}(x,y,\alpha)$
by
\small
\begin{equation}\label{cellformula}
\left(\begin{array}{l}
X_x=C_J(y,(\alpha_1,\alpha_2))\ \wedge \\
\displaystyle\bigwedge_{1\leq i\leq |J|} \forall t \left( C_{\delta_i}(t,y_i,\alpha_{1i},\alpha_{2i})\to \left(f_x(t)=\alpha_{3i}+a_i\frac{\ord(t-y_i))-k_{\delta_i}}{n_{\delta_i}}\right)\right)\end{array}\right).\\
\end{equation}
\normalsize
Roughly, this formula states that the set $X_x$ can be decomposed into finitely many cells satifying condition \ref{eq:preparation}. Define the set $W_N$ by 
\begin{equation}\label{WN}
W_N:=\{(J,a): J\subseteq E_{K,N}, a\in \ZZ^{|J|}, |a_i|<N \text{ for all } 1\leq i\leq |J|\},
\end{equation}
and consider the set of formulas 
\[\Sigma(x):=\left\{ \bigwedge_{(J,a)\in W_N} \neg (\exists y\in K^{|J|}\exists \alpha\in \Gamma_{K}^{3|J|}\phi_{J,a}(x,y,\alpha)) \mid N\in\mathbb{N}^*\right\}.
\]
By cell decomposition (Proposition \ref{prop.celldecomp}), $\Sigma(x)$ is inconsistent. Thus, by compactness, there exists a finite subset $\Sigma_0(x)$ which is inconsistent. Note that for every finite subset $\Sigma_0(x)\subseteq\Sigma(x)$, one can find a positive integer $N$ such that 
\[\bigwedge_{(J,a)\in W_N} \neg (\exists y\in K^{|J|}\exists \alpha\in \Gamma_{K}^{3|J|}\phi_{J,a}(x,y,\alpha)) \models \Sigma_0(x).
\end{equation*}

This implies that there must exist $N>0$ such that for every $s\in \Pi_S(X)$  
\[(K,\Gamma_K)\models \bigvee_{(J,a)\in W_N} \exists y\in K^{|J|}\exists \alpha\in \Gamma_{K}^{3|J|}\phi_{J,a}(s,y,\alpha)) 
\]
which is exactly what we wanted. 
\end{proof}

Let $N$ be given by the claim and $W_N$ as defined in \ref{WN}. Since $W_N$ is finite, one can put a total ordering $\lessdot$ on it. We will also put an alternative ordering $\lhd$ on the value group $\Gamma_K$ defined by :
\[x \lhd y \Leftrightarrow
(0 \leqslant x <y) \vee (0< x \leqslant -y) \vee (0< -x < y) \vee (0< -x < -y).\]
This produces a total ordering on $\Gamma_K$ which can be extended to $\Gamma_K^k$ lexicographically. We also denote this extension by $\lhd$. The important property of the order $\lhd$ is that every definable set of $\Gamma_K$ has a $\lhd$-smallest element (for a proof of this, see the appendix, in particular \ref{preswellorder}).

\

Now consider the map \[\sigma: S \to W_N \times (\Gamma_K)^{\leq 3|W_N|}: s \mapsto (\sigma_1(s), \sigma_2(s)),\]where $\sigma_1, \sigma_2$ are defined as follows:
\begin{itemize}
\item put $\sigma_1(s)= (J,a)$, if $(J,a)$ is the $\lessdot$-smallest element of $W_N$ such that  
$$(K,\Gamma_K)\models \exists y\in K^{|J|}\exists \alpha\in \Gamma_{K}^{3|J|}\phi_{J,a}(s,y,\alpha).$$
The claim ensures the existence of at least one such $(J,a)$ in $W_N$.    
\item let $\phi_2(s)$ be the $\lhd$-smallest tuple $\alpha\in \Gamma_K^{3|J|}$ such that 
$$(K,\Gamma_K)\models \exists y\in K^{|J|}\phi_{\sigma_1(s)}(s,y,\alpha).$$
\end{itemize}
It is clear that the function $\sigma$ will be definable, once we fix a representation for the finite index set $W_N$. From this information one can then reconstruct the $K$-cell decomposition for $X$ and the linear functions satisfying \ref{eq:preparation}, in the following way. Let $S_0:=\Pi_S(X)$ and for $\lambda\in \sigma_1(S_0)$ let $S_{\lambda}$ be the set $\{s \in S_0 \mid \sigma_1(s)=\lambda\}$. For $\lambda=(J,a)\in \sigma_1(S_0)$, $1\leq j\leq |J|$ and $i=1,2,3$ let $\alpha_{\lambda ji}: S_\lambda\to \Gamma_K$ be defined by 

\[\sigma_2(s)=\left(
\begin{array}{l}
\alpha_{\lambda 11}(s),\ldots,\alpha_{\lambda j1}(s),\ldots, \alpha_{\lambda |J|1}(s),\\
\alpha_{\lambda 12}(s),\ldots,\alpha_{\lambda j2}(s),\ldots, \alpha_{\lambda |J|2}(s),\\
\alpha_{\lambda 13}(s),\ldots,\alpha_{\lambda j3}(s),\ldots, \alpha_{\lambda |J|3}(s),
\end{array}\right).\] 

For $\lambda=(J,a)\in W_N$, and $s\in S_\lambda$ define 

\[\Sigma_{\lambda,s}:= \{y\in K^{|J|}|(K,\Gamma_K)\models\phi_{\sigma_1(s)}(s,y,\alpha)\},\]

which is non-empty by definition of $\sigma_1$. With this we produce a finite partition $X=\bigcup\{X_\lambda|\lambda=(J,a)\in \sigma_1(S_0)\}$ where 
\small
\[X_\lambda =\left\{(s,x)\in S_{\lambda}\times K\left|\forall (c_1,\ldots,c_{|J|})\in \Sigma_{\lambda,s}:\bigvee_{1\leq j\leq |J|}C_{\lambda}(x,c_j,\alpha_{\lambda 1j},\alpha_{\lambda 2j};s)\right\}\right..
\]
\normalsize
By construction, for each $\lambda=(J,a)\in \sigma_1(S_0)$, each $1\leq j\leq |J|$, each $s\in S_\lambda$, each $\delta_j\in J$ ($1\leq j\leq|J|$) and all $(c_1,\ldots,c_{|J|})\in \Sigma_{\lambda,s}$ the following holds \small
\[\forall t\in K \left(C_{\delta_j}(t,c_j,\alpha_{\lambda 1j},\alpha_{\lambda 2j};s) \rightarrow \left(f_s(t)=\alpha_{\lambda 3j}+a_j\frac{\ord(t-c_j)-k_{\delta_j}}{n_{\delta_j}}\right)\right).
\]
\normalsize
 
\end{proof}

Our proof also shows: 

\begin{thm}[$K$-cell decomposition for $P$-minimal structures]
Let $K$ be a $P$-minimal $\Lm$-structure and $X \subseteq S \times K$ be a definable set. There exists a $K$-cell decomposition $\{(\Sigma_i)_{i}, (C_{\delta_{ij}})_{i,j})\}$ of $X$.
\end{thm}
\begin{proof} Ommiting the function preparation part in the previous argument gives the $K$-cell decomposition. $P$-minimality is used to prove an analogue of claim \ref{claim:compact}. 
\end{proof}

We now present the preparation theorem for definable subsets of the form $X\subseteq S\times \Gamma_K$. We will use the following theorem of Cluckers \cite{clu-presb03}.
\begin{thm}\label{thm:semialgpres}
Let $(K, \Lm)$ be a $P$-minimal field with $P$-minimal theory. 
\item For any $\Lm$-definable set $X \subseteq (K^{\times})^m$, the set
\[\ord(X):=\{(\ord x_1, \ldots, \ord x_m) \mid (x_1, \ldots, x_m) \in X\}\]
is $\Lm_{\text{Pres}}$-definable.
\item Let $S \subseteq \Gamma_K^m$ be a Presburger-definable set. Then the set
\[\{\ord^{-1}(S):= \{ (x_1, \ldots, x_m) \in X \mid \ord x \in S\}\]
is $\Lring$-definable.
\end{thm}

\begin{prop}\label{prop:partialcd2}
%Let $(K, \Gamma_K, \Lm_2)$ be a relatively P-minimal structure.
%Let $S \subseteq K^m \times (\Gamma_K)^{m'}$, 
Let $X \subseteq S \times \Gamma_K$ be a definable set, and $f: X \to \Gamma_K$ a definable function. There exists a finite decomposition of $X$ into cells $C$ of the form
\[C = \left\{(x,\gamma) \in D \times \Gamma_K \ \left| \ \begin{array}{l}\alpha(x)\ \square_1 \ \gamma \ \square_2 \  \beta(x)\quad \text{and}\quad \gamma \equiv n_0 \mod n \end{array} \right\}\right.,\]
where $D$ is an definable subset of $S$, $\alpha, \beta: S \to \Gamma_K$ are definable functions, and $n_0,n \in \NN$. The partition can be taken such that on each part $C$, there exists a constant $a_{C} \in \ZZ$ and a definable function $\delta: D \to \Gamma_K$, such that for all $(x,\gamma) \in C$,
\[f(x,\gamma) = a_C \left(\frac{\gamma -n_0}{n}\right) + \delta(x).\]
 \end{prop}
\begin{proof}
The proof is similar to the proof of the previous theorem. By relative $P$-minimality, we know that for every $s \in \Pi_X(S)$, the fiber $X_s$ is a Presburger definable set. Cluckers \cite{clu-presb03} obtained a cell decomposition theorem for Presburger structures. Applying this to the sets $X_s$, yields that each
 $X_s$ can be partitoned into a finite union of cells of the form
\[C_s := \{ \gamma \in K \mid  \alpha_s\ \square_1 \ \gamma \ \square_2 \  \beta_s\quad \text{and}\quad \gamma \equiv n_0 \mod n_s\},\]
where $\alpha_s, \beta_s \in \Gamma_K$ and $n_s \in \NN$ are constants depending on $s$. Also note that for any $s \in \Pi_X(S)$, the graph of the function $f_s(\gamma):= f(s,\gamma)$ will be a Presburger set, by the assumption of relative $P$-minimality. Indeed, $G_s:=\{(x,\gamma)\in K\times \Gamma_K: f_s(\ord(x))=\gamma)\}$ is $\Lm_{ring,2}$-definable by relative $P$-minimality. Therefore the set $G_s':=\{(x,y)\in K^2: (x,\ord(y))\in G_s\}$ is semi-algebraic. But 
\[Graph(f_s)=\{(\ord(x),\ord(y))\in \Gamma^2: (x,y)\in G_s'\},\]
which by Theorem \ref{thm:semialgpres} is Presburger definable. This means that each $f_s$ is a Presburger definable funtion, and hence  must be piecewise linear (with coefficients in $\QQ$). In particular, the above partition can be taken such that on each $C_s$, there exist constants $a_s \in \ZZ, \delta_s \in \gamma_K$ such that for all $\gamma \in C_s$, we have that
\[f_s(\gamma) = a_s \left(\frac{\gamma -n_0}{n_s}\right) + \delta_s.\]

Our claims follow then from a similar argument as the one presented in the proof of Theorem
\ref{thm:partialprep}. Since no centers are required in this case, the cell decomposition we obtain is the classical one.
\end{proof}


