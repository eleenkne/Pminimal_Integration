% !TEX root = nov.tex
\section{Integration}
In this section, $K$ denotes a $p$-adic field, so the value group $\Gamma_K$ will just be $\Z$.
We first recall the following Definition and Lemma from \cite{Clu-Gor-Hal-14}.
\begin{defn}Let $D \subseteq \ZZ^r$ be a Presburger set, $S$ an $\Lm_2$-definable set.
\item A function $f: D \to \AA_{q_K}$ is called Presburger constructible if it is contained in the $\AA_{q_K}$-algebra generated by Presburger functions $D\to \ZZ$, and functions $D\to \AA_{q_K}: x \mapsto q_K^{\beta(x)}$, where $\beta:D \to \ZZ$ is a Presburger function.
\item A function $f:X\subseteq S \times \ZZ^r$ is called Presburger constructible over $S$ if  the restrictions $f_s: X_s \to \AA_{q_K}$ are (uniformly) Presburger constructible for all $s \in S$. 

\end{defn}
\begin{lem}\label{lemma:presburgerloci} Let $S$ be an $\Lm_2$-definable set. 
Let $f:Y\subseteq S\times \ZZ^r\to \AA_{q_K}$ be a function which is Presburger-constructible over $S$.
\item There exist $\Lm_2$-constructible functions $h_i$, such that {\color{red} If we are going to use this, need to explain these notations!!}
\begin{align*}
\text{Int}(f,S) &= Z(h_1),\\
\text{Bdd}(f,S) &= Z(h_2),\\
\text{Iva}(f,S) &= Z(h_3).
\end{align*}
\item There exists a function $g: Y \to \AA_{q_K}$ that is Presburger-constructible over $S$, such that $\Int(g,S) = S$ and $f(s,y) = g(s,y)$ whenever $s \in \text{Int}(f,S)$.
\end{lem}
\begin{thm}
Assume that for every $s \in S$, the function $f: S\times K^m \to \QQ$ is constructible in a relatively P-minimal structure. If $\int f(s,x)|dx|$ is finite for all $s \in S$, then there exists a constructible function $g: S \to \QQ$, such that
\[g(s) = \int_{X_s} f(s,x)|dx|.\]
\end{thm}
\begin{proof}
Since the result for general $m$ can be obtained by iteration, we may assume that $m =1$. A general constructible function has the form
\[f(s,x) = \sum_{i=1}^r a_i q_K^{f_{i0}(s,x)} \prod_{j=1}^{r'}f_{ij}(s,x),\]
where the $f_i, f_{ij}$ are definable functions $X \to \ZZ$ (where $\ZZ$ equals the value group $\Gamma_K$), and $a_i \in \ZZ\left[q, \frac1q, (\frac{1}{1-q^{-i}})_{i\in \NN\backslash\{0\}}\right].$

Now put $\gamma = (\gamma_{ij})_{i,j}$ and consider the set
\[\text{GR}(f):= \{(s,\gamma, x)\in S\times \Gamma_K^{(r'+1)r}\times K \mid  \gamma_{ij} = f_{ij}(s,x)\},\]
which is a permutated version of the combined graphs of the functions generating $f$. We will partition this set in cells, using the above cell decomposition results. It is easy to see that an iteration of Theorem \ref{thm:partialcd1} and Proposition \ref{prop:partialcd2} yields the following.

Put $R = r(t'+1)$. To ease the notation somewhat, we will sometimes renumber $\gamma =(\gamma_i)_{i=1\ldots R}$, where the correspondence is $\gamma_{ij}:= \gamma_{(i-1)(r'+1)+j+1}$.
Write $\langle \gamma\rangle_i := (\gamma_1, \ldots, \gamma_{i})$.
There exists a partition of $\text{GR}(f)$ into a finite union of cells $A$ of the following form. 
For each $A$, one has a definable set $\tilde{S}_A \subseteq S$, and for $1 \leqslant i \leqslant R+1$, a tuple of definable functions
\[\alpha_i: \tilde{S}_A \times \Gamma_K^{i-1} \to \Gamma_K: (s, \langle \gamma\rangle_{i-1}) \mapsto \alpha_i(s, \langle \gamma\rangle_{i-1}),\]
and similarly defined functions $\beta_i$. Put $\alpha := \alpha_{R+1}, \beta:= \beta_{R+1}$.  One also has a (not necessarily definable) function $c: \tilde{S}_A \times \Gamma_K^{R}\to K$.
For each $i =1, \ldots R$, let $\lambda_i =(\square_{i,1}, \square_{i,2},k,N) \in P_{\Gamma}$. We recall the notation 
\[
%B_{\lambda_i}(\alpha_i,\beta_i,\gamma_i):= \left[\begin{array}{l} \alpha_i(s,\langle \gamma\rangle_{i-1})\ \square_{i,1} \ \gamma_i \ \square_{i,2} \ \beta_i(s,\langle \gamma\rangle_{i-1}),  \ \ \ \wedge \\
% \gamma_i \equiv k_i\mod N \end{array}\right].
B_{\lambda_i}(\alpha_i,\beta_i,\gamma_i):=  [\alpha_i(s,\langle \gamma\rangle_{i-1})\ \square_{i,1} \ \gamma_i \ \square_{i,2} \ \beta_i(s,\langle \gamma\rangle_{i-1}) \ \wedge \
 \gamma_i \equiv k_i\mod N] .
\]
If we let $B$ be the set \[
B:= \left\{(s,\gamma) \in \tilde{S}_A \times \Gamma_K^R \ \left| \ \bigwedge_{i=1}^R B_{\lambda_i}(\alpha_i,\beta_i,\gamma_i)\right\}\right.,\]
then the partition of $\text{GR}(f)$ will consist of cells of the form
\[A:=\left\{(s, \gamma, x) \in B \times K \left| \begin{array}{l} 
\alpha(s,\gamma) \ \square_{1}\ \ord(x -c(s,\gamma)) \ \square_{2}\ \beta(s, \gamma),\\
\ord(x -c(s,\gamma)) \equiv k \mod N,\\ \ac_M(x -c(s,\gamma, )) = \xi
\end{array}
\right\}\right.,\]
for some $(\square, k, N, M, \xi) \in P_K$.

%VANAF HIER HERWERKEN
This partitioning can now be used to compute the given integral. Let $\mu$ denote the usual Haar measure (i.e., normalized such that $\mu(R_K) =1$. ) We get the following:
\begin{eqnarray*}
 \int_{X_s} {f(s,x)}|dx| &=& \sum_\gamma \left[\left(\sum_{i=1}^r a_i q_K^{\gamma_{i0}} \prod_{j=1}^{r'}\gamma_{ij}\right)\cdot \mu(\{x\in X_s \mid \bigwedge_{ij} f_{ij}(s,x) = \gamma_{ij}\}\right]\\ 
 &=& \sum_{\{A \mid s \in \tilde{S}_A\}} \left[ \sum_{\gamma \in B_s} \left(\sum_{i=1}^r a_i q_K^{\gamma_{i0}} \prod_{j=1}^{r'}\gamma_{ij}\right) \cdot \mu(A_{s,\gamma})\right].
\end{eqnarray*}
Let us now compute the measure of a fiber $A_{s,\gamma}$. %We will extend the notation $\langle \ldots\rangle_{i}$ introduced above, and write $\langle D\rangle_{i}$ for the projection of the elements of $D$ onto their first $i$ co\"ordinates.  For each $\langle x\rangle_{i-1} \in \langle A_{s,\gamma}\rangle_{i-1}$, write 
%\begin{align*}\Phi_{s,\gamma}(\langle x\rangle_{i-1})&:= \{ x_{i} \in K \mid x_i \text{ satisfies } \phi_{i1}(c_i) \wedge \phi_{i2}(c_i)\}\\
%\Phi^{0}_{s,\gamma}(\langle x\rangle_{i-1})&:= \{ x_{i} \in K \mid x_i \text{ satisfies } \phi_{i1}(0) \wedge \phi_{i2}(0)\}
%\end{align*}
Put $u =x - c(s, \gamma)$. Since the Haar measure is translation invariant, we have $|dx|=|du|$. Let $\hat{A}$ denote the set
\[\hat{A}:=\left\{(s, \gamma, u) \in B \times K \left| \begin{array}{l} 
\alpha(s,\gamma) \ \square_{1}\ \ord u \ \square_{2}\ \beta(s, \gamma),\\
\ord u \equiv k \mod N,\\ \ac_M(u) = \xi
\end{array}
\right\}\right.,\]
We get
\begin{eqnarray*}
\mu(A_{s,\gamma}) &=& \int_{A_{s,\gamma}}|dx|\\
&=& \int_{\hat{A}_{s,\gamma}}|du|
\\
&=&q_K^{-(k+M)}\sum_{\tau \in T} (q_K^{-N})^\tau,
\end{eqnarray*}
where $T$ is the set $T:=\{\tau \in \Gamma_K \mid \alpha(s,\gamma ) \ \Box_{1}\ k + \tau N\ \square_{2}\ \beta(s,\gamma)\}$.
Because of our assumptions, the set $A_{s,\gamma}$ must have finite measure, which is only possible if $\square_{1}$ denotes $<$.
{\color{red}
(Note: this is a property of the cell $A$, making it a definable condition)
} 

We get the following results for this sum. 
%If possible, we will suppress the variables and just write $\alpha_m, \beta_m, \ldots$ to keep things readable. 
Put $\tilde{\alpha}:= \lfloor\frac{\alpha -k}{N}\rfloor+1$, and $\tilde{\beta}:= \lceil\frac{\beta -k}{N}\rceil-1$ (clearly these are still $\Lm_2$-definable functions). 

\begin{description}
\item[If $\square_{1}$ denotes \emph{no condition}]
\[\sum_{\tau \in T} (q_K^{-N})^\tau = \frac{q_K^{-N\tilde{\alpha}}}{1-q_K^{-N}} \]
\item[If $\square_{1}$ denotes <]
\[\sum_{\tau \in T} (q_K^{-N})^\tau = \frac{q_K^{-N\tilde{\alpha}}-q_K^{-N\tilde{\beta}}}{1-q_K^{-N}} \]
\end{description}
These are both constructible functions, so there exists a constructible function $h(s,\gamma)$ such that \[h(s,\gamma) = \mu(A_{s, \gamma}).\]
Note that by (an interation of) proposition \ref{prop:partialcd2}, we may assume that the variables $\gamma_i$ only occur in the functions $\alpha_i$, $\beta_i$ in a {\color{red} linear} way (possibly after refining the cell decomposition). Hence, the problem is reduced to showing that a sum of the form 
\[\sum_{\gamma \in B_s} \left(\sum_{i=1}^r a_i q_K^{\color{red}\mu_i\gamma_{i0} + \nu_i(s)} \prod_{j=1}^{r'}\gamma_{ij}\right)\]
%$\sum_{\gamma \in \Pi_{\gamma}(A_s)} q_K^{a\gamma + \delta(s)} $ 
can be computed uniformly.
Let $F: B\to \AA_{q_K}$ be the function defined by
\[F(s,\gamma) = \left\{\begin{array}{ll}{\sum_{i=1}^r a_i q_K^{\mu_i\gamma_{i0} + \nu_i(s)} \prod_{j=1}^{r'}\gamma_{ij}} & \text{if } (s,\gamma)\in B\\ 0 & \text{otherwise}.
\end{array}.\right.\]
This function is Presburger constructible over $S$, so we can apply
Lemma \ref{lemma:presburgerloci}. This lemma implies that the locus of integrability can be defined uniformly as the zero locus of an $\Lm$-constructible function $h: S\to \AA_{q_K}$. Also, we can find a function $G$ with $\text{Int}(G,S)= S$ and $G(s,\gamma) = F(s,\gamma)$ when $s \in \text{Int}(F,S)$. Because of this, we may as well assume that $\text{Int}(F,S)=S$.
{\color{red} And then I think we can just \emph{quote/adapt} Theorem-Definition 4.5.1 in \cite{clu-loe-08}. The difference is that the dependence on the parameter space is $\Lm_2$-definable rather than Presburger definable, but I don't think that changes anything in the proof.}
\end{proof}
%\begin{defn}

%\end{defn}
