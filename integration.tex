% !TEX root = nov.tex
\section{Integration}
In this section, $K$ denotes a $p$-adic field, so the value group $\Gamma_K$ will just be $\Z$. Two types of integrals will appear. When integrating over (subsets of) $\cO_K^m$, the Haar measure $\mu$ is used. When integrating over $\Z^n$, we use the counting measure.

We first recall the following Definition and Theorem from \cite{Clu-Gor-Hal-14}.
\begin{defn}Let $D \subseteq \ZZ^r$ be a Presburger set, $S$ a definable set.
\item A function $f: D \to \AA_{q_K}$ is called Presburger constructible if it is contained in the $\AA_{q_K}$-algebra generated by Presburger functions $D\to \ZZ$, and functions $D\to \AA_{q_K}: x \mapsto q_K^{\beta(x)}$, where $\beta:D \to \ZZ$ is a Presburger function.
\item A function $f:X\subseteq S \times \ZZ^r$ is called Presburger constructible over $S$ if  the restrictions $f_s: X_s \to \AA_{q_K}$ are (uniformly) Presburger constructible for all $s \in S$. 
\item Given a function $h:D\to \AA_{q_K}$, we let $Z(h)$ denote the zero locus of $h$.
\end{defn}
\begin{def-theorem}[\cite{Clu-Gor-Hal-14}, Theorems 3.1.3 and 3.1.5]\label{thm:presburgerloci} Let $S$ be a definable set, 
and $f:Y\subseteq S\times \ZZ^r\to \AA_{q_K}$ a function which is Presburger-constructible over $S$. Define the following sets:
\begin{align*}
\text{Iva}(f,S)&:=\{s \in S \mid f(s, \cdot) \text{ is identically zero on } Y_s\}\\
\text{Int}(f,S) &:=\{s \in S \mid f(s, \cdot) \text{ is measurable and integrable on } Y_s \}
\end{align*}
\item There exist $\Lm_2$-constructible functions $h_i: S \to \AA_{q_K}$, such that 
\[\text{Iva}(f,S) = Z(h_1), \text{\quad and \quad }
\text{Int}(f,S) = Z(h_2).\]
%\text{Bdd}(f,S) &= Z(h_2),\\
%.
%\end{align*}
\item There exists a function $g: Y \to \AA_{q_K}$, Presburger-constructible over $S$, such that $\Int(g,S) = S$ and $f(s,y) = g(s,y)$ whenever $s \in \text{Int}(f,S)$.
\end{def-theorem}
We will also need the following result
\begin{thm}\label{thm:presburger-int}
Let $S$ be a definable set and $f: Y \subseteq S \times \ZZ^r \to \AA_{q_K}$ a function which is Presburger-constructible over $S$. If $\text{Int}(f,S) =S$, then there exists an $\Lm_2$-constructible function $g:S \to \AA_{q_K}$, such that for all $s \in S$,
\[ g(s) = \int_{Y_s} f(s,t)|dt|.\]  
\end{thm}
\begin{proof}
This is a reformulation of Theorem-Definition 4.5.1 in \cite{clu-loe-08}. The original theorem is for the case where $S$ is Presburger-definable, but the same proof applies in this context.
\end{proof}
We are now ready to state and prove our main theorem:
\begin{thm}\label{thm:integration}
Let $(K, \Z)$ be a relatively $P$-minimal structure, $S$ a definable set and $f: X \subseteq S\times K^m \to \AA_{q_K}$ a constructible function.  There exists a constructible function $g: S \to \AA_{q_K}$, such that
\[g(s) = \int_{X_s} f(s,x)|dx|,\]
whenever $s \in \text{Int}(f,S)$.
\end{thm}
\begin{proof}
Since the result for general $m$ can be obtained by iteration, we may assume that $m =1$. A general constructible function has the form
\[f(s,x) = \sum_{i=1}^r a_i q_K^{f_{i0}(s,x)} \prod_{j=1}^{r'}f_{ij}(s,x),\]
where the $f_{ij}$ are definable functions $X \to \ZZ$, and $a_i \in \AA_{q_K}.$
Now put $\gamma = (\gamma_{ij})_{i,j}$ and consider the set
\[\text{GR}(f):= \{(s,\gamma, x)\in S\times \Gamma_K^{(r'+1)r}\times K \mid  \gamma_{ij} = f_{ij}(s,x)\},\]
which is a permutated version of the combined graphs of the functions generating $f$. We will partition this set in cells, using the cell decomposition results from the previous section. It is easy to see that an iteration of Theorem \ref{thm:partialprep} and Proposition \ref{prop:partialcd2} yields the following.

Put $R = r(r'+1)$. To ease the notation somewhat, we will sometimes renumber $\gamma =(\gamma_i)_{i=1\ldots R}$, where the correspondence is $\gamma_{ij}:= \gamma_{(i-1)(r'+1)+j+1}$.
Write $\langle \gamma\rangle_i := (\gamma_1, \ldots, \gamma_{i})$.
There exists a partition of $\text{GR}(f)$ into a finite union of cells $A$ of the following form. 
For each $A$, one has a definable set $\tilde{S}_A \subseteq S$, and for $1 \leqslant i \leqslant R+1$, a tuple of definable functions
\[\alpha_i: \tilde{S}_A \times \Gamma_K^{i-1} \to \Gamma_K: (s, \langle \gamma\rangle_{i-1}) \mapsto \alpha_i(s, \langle \gamma\rangle_{i-1}),\]
and similarly defined functions $\beta_i$. Put $\alpha := \alpha_{R+1}, \beta:= \beta_{R+1}$.  One also has a (not necessarily definable) function $c: \tilde{S}_A \times \Gamma_K^{R}\to K$.
For each $i =1, \ldots R$, let $\lambda_i =(\square_{i,1}, \square_{i,2},k,N) \in P_{\Gamma}$. We recall the notation 
\[
%B_{\lambda_i}(\alpha_i,\beta_i,\gamma_i):= \left[\begin{array}{l} \alpha_i(s,\langle \gamma\rangle_{i-1})\ \square_{i,1} \ \gamma_i \ \square_{i,2} \ \beta_i(s,\langle \gamma\rangle_{i-1}),  \ \ \ \wedge \\
% \gamma_i \equiv k_i\mod N \end{array}\right].
B_{\lambda_i}(\alpha_i,\beta_i,\gamma_i):=  [\alpha_i(s,\langle \gamma\rangle_{i-1})\ \square_{i,1} \ \gamma_i \ \square_{i,2} \ \beta_i(s,\langle \gamma\rangle_{i-1}) \ \wedge \
 \gamma_i \equiv k_i\mod N] .
\]
If we let $B$ be the set \[
B:= \left\{(s,\gamma) \in \tilde{S}_A \times \Gamma_K^R \ \left| \ \bigwedge_{i=1}^R B_{\lambda_i}(\alpha_i,\beta_i,\gamma_i)\right\}\right.,\]
then the partition of $\text{GR}(f)$ will consist of cells of the form
\[A:=\left\{(s, \gamma, x) \in B \times K \left| \begin{array}{l} 
\alpha(s,\gamma) \ \square_{1}\ \ord(x -c(s,\gamma)) \ \square_{2}\ \beta(s, \gamma),\\
\ord(x -c(s,\gamma)) \equiv k \mod N,\\ \ac_M(x -c(s,\gamma, )) = \xi
\end{array}
\right\}\right.,\]
for some $(\square, k, N, M, \xi) \in P_K$.

This partitioning can now be used to compute the given integral. Let $\mu$ denote the usual Haar measure (i.e., normalized such that $\mu(cO_K) =1$. ) We get the following:
\begin{eqnarray*}
 \int_{X_s} {f(s,x)}|dx|
 &=& \sum_{\delta \in \text{Im}(f_s)} \delta \cdot \mu\{x\in X_s \mid f_s(x) = \delta\}\\
 &=& \sum_\gamma  \left[\left(\sum_{i=1}^r a_i q_K^{\gamma_{i0}} \prod_{j=1}^{r'}\gamma_{ij}\right)\cdot \mu(\{x\in X_s \mid \bigwedge_{ij} f_{ij}(s,x) = \gamma_{ij}\}\right]\\ 
 &=& \sum_{\{A \mid s \in \tilde{S}_A\}} \left[ \sum_{\gamma \in B_s} \left(\sum_{i=1}^r a_i q_K^{\gamma_{i0}} \prod_{j=1}^{r'}\gamma_{ij}\right) \cdot \mu(A_{s,\gamma})\right].
\end{eqnarray*}
Let us now compute the measure of a fiber $A_{s,\gamma}$. %We will extend the notation $\langle \ldots\rangle_{i}$ introduced above, and write $\langle D\rangle_{i}$ for the projection of the elements of $D$ onto their first $i$ co\"ordinates.  For each $\langle x\rangle_{i-1} \in \langle A_{s,\gamma}\rangle_{i-1}$, write 
%\begin{align*}\Phi_{s,\gamma}(\langle x\rangle_{i-1})&:= \{ x_{i} \in K \mid x_i \text{ satisfies } \phi_{i1}(c_i) \wedge \phi_{i2}(c_i)\}\\
%\Phi^{0}_{s,\gamma}(\langle x\rangle_{i-1})&:= \{ x_{i} \in K \mid x_i \text{ satisfies } \phi_{i1}(0) \wedge \phi_{i2}(0)\}
%\end{align*}
Put $u =x - c(s, \gamma)$. Since the Haar measure is translation invariant, we have $|dx|=|du|$. Let $\hat{A}$ denote the set
\[\hat{A}:=\left\{(s, \gamma, u) \in B \times K \left| \begin{array}{l} 
\alpha(s,\gamma) \ \square_{1}\ \ord u \ \square_{2}\ \beta(s, \gamma),\\
\ord u \equiv k \mod N,\\ \ac_M(u) = \xi
\end{array}
\right\}\right.,\]
We get
\begin{eqnarray*}
\mu(A_{s,\gamma}) &=& \int_{A_{s,\gamma}}|dx|\\
&=& \int_{\hat{A}_{s,\gamma}}|du|
\\
&=& \sum_{\tau \in T}\mu(\xi \pi_K^{k + \tau N} (1 + \pi^{M}\cO_K)),\\
&=&|\xi|q_K^{-(k+M)}\sum_{\tau \in T} (q_K^{-N})^\tau,
\end{eqnarray*}
where $\pi_K$ is a uniformizing element of $K$ and $T$ is the set \[T:=\{\tau \in \Gamma_K \mid \alpha(s,\gamma ) \ \Box_{1}\ k + \tau N\ \square_{2}\ \beta(s,\gamma)\}.\]
Because of our assumptions, the set $A_{s,\gamma}$ must have finite measure, which is only possible if $\square_{1}$ denotes $<$.
(This is a property of the cell $A$, making it a definable condition)
 
We get the following results for this sum. 
%If possible, we will suppress the variables and just write $\alpha_m, \beta_m, \ldots$ to keep things readable. 
Put $\tilde{\alpha}:= \lfloor\frac{\alpha -k}{N}\rfloor+1$, and $\tilde{\beta}:= \lceil\frac{\beta -k}{N}\rceil-1$ (clearly these are still definable functions). 

\begin{description}
\item[If $\square_{1}$ denotes \emph{no condition}]
\begin{equation}\label{eq:nocond}\sum_{\tau \in T} (q_K^{-N})^\tau = \frac{q_K^{-N\tilde{\alpha}}}{1-q_K^{-N}} \end{equation}
\item[If $\square_{1}$ denotes <]
\begin{equation}\label{eq:ineq}\sum_{\tau \in T} (q_K^{-N})^\tau = \frac{q_K^{-N\tilde{\alpha}}-q_K^{-N\tilde{\beta}}}{1-q_K^{-N}} \end{equation}
\end{description}
These are both constructible functions, so there exists a constructible function $h(s,\gamma)$ such that \[h(s,\gamma) = \mu(A_{s, \gamma}).\]
Hence, returning to the computation of $\int_{X_s} f(s,x)|dx|$, we need to compute a sum of the form
\[\sum_{\gamma \in B_s}\left[ \left(\sum_{i=1}^r a_i q_K^{\gamma_{i0}} \prod_{j=1}^{r'}\gamma_{ij}\right) \cdot h(s,\gamma)\right]\]
To compute this sum over gamma, we need to specify how $h(s,\gamma)$ depends on $\gamma$. 
By (an interation of) proposition \ref{prop:partialcd2}, we may assume that the variables $\gamma_i$ only occur in the functions $\alpha_i$, $\beta_i$ (in \eqref{eq:ineq} and \eqref{eq:nocond} ) in a { linear} way (possibly after refining the cell decomposition). Hence, the problem is reduced to showing that a sum of the form 
\[\sum_{\gamma \in B_s} \sum_{i=1}^r \left(a_i q_K^{\nu_i(s)}q_K^{b_i(\gamma)} \prod_{j=1}^{R}\gamma_{j}^{a_{ij}}\right),\]
%$\sum_{\gamma \in \Pi_{\gamma}(A_s)} q_K^{a\gamma + \delta(s)} $ 
where the $b_i$ are Presburger definable linear functions, %(i.e., of the form $\sum_{j=1}^R{\mu_{ij}\left(\frac{\gamma_{i}-k_i}{N}\right)}$)
$a_{ij} \in \mathbb{N}$ and $a_i \in \AA_{q_K}$, can be computed uniformly.
Let $F: B\to \AA_{q_K}$ be the function defined by
\[F(s,\gamma) = \left\{\begin{array}{ll}{\sum_{i=1}^r \left(a_i q_K^{\nu_i(s)}q_K^{b_i(\gamma)} \prod_{j=1}^{R}\gamma_{j}^{a_{ij}}\right)} & \text{if } (s,\gamma)\in B\\ 0 & \text{otherwise}.
\end{array}.\right.\]
This function is Presburger constructible over $S$, so we can apply
Theorem-Definition \ref{thm:presburgerloci} to conclude that its locus of integrability $\text{Int}(F,S)$ can be defined uniformly as the zero locus of an $\Lm$-constructible function $h: S\to \AA_{q_K}$. 
This lemma also implies that one can find a function $G(s,\gamma)$ with $\text{Int}(G,S)= S$ and $G(s,\gamma) = F(s,\gamma)$ whenever $s \in \text{Int}(F,S)$. Because of this, we may as well assume that $\text{Int}(F,S)=S$. Applying Theorem \ref{thm:presburger-int} to $G(s,\gamma)$ concludes the proof.
\end{proof}
%\begin{defn}

%\end{defn}
