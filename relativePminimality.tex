% !TEX root = nov.tex
\section{Relative $P$-minimality} \label{sec:relativity}

Valued fields constitute an example of structures in which it is quite natural to work in a multi-sorted setting adding auxiliary sorts for the value group $\Gamma_K$ and for the residue field $k_K$. When the field sort has a very tame structure ($p$-adic fields, algebraically closed valued fields), it often imposes not only good properties on all sorts but also a tame interaction between them. Nevertheless, it is not known if minimality conditions on the field sort (as $P$-minimality of $C$-minimality) are strong enough to force these interactions to be tame along the sorts involved. As an example, preparation or cell-decomposition theorems are not known to be true. The notion of ``relative minimality'' here suggested arises as an attempt to deal with this issue. Similar assumption have been made in other contexts.  In its more abstract format, the following is general definition of relative minimality due to Cluckers and Halupszoc:

\begin{defn}(Relative minimality)\label{def:relmini}
Let $\Lm_B$ be a language with one main sort and possibly some other sorts which are all called auxiliary sorts. Let $\Lm$ be an expansion of $\Lm_B$ and $K$ be an $\Lm$-structure. Say that $K$ is relatively $\Lm_B$-minimal (relative to the auxiliary sorts) if for each Cartesian product $S$ of
universes of auxiliary sorts of $K$, the following holds:
any $L$-definable set $X\subseteq K\times S$ is already $\Lm_B$-definable.
\end{defn}

If we apply this definition to the two-sorted language $\Lringtwo$ we seem to get a stronger notion than relative $P$-minimality. For an $\Lm_2$-structure $(K,\Gamma_K)$, being $\Lringtwo$-minimal implies that all definable subsets $X\subseteq K\times\Gamma_K^d$ are $\Lm_{ring,2}$-definable, while relative $P$-minimality only gives this when $d=1$. The next proposition shows that relative $P$-minimal structures are actually $\Lringtwo$-minimal. 

\begin{prop}\label{l=1} Every relative $P$-minimal $\Lm_2$-structure is $\Lm_{ring,2}$-minimal. 
\end{prop}

\begin{proof} We prove that all definable subsets $X\subseteq K\times \Gamma_K^d$ are $\Lringtwo$-definbale. By induction on $d\geq 1$ we prove the following two claims:
\begin{enumerate}
\item[$A_d$:] Any $\Lm_2$-definable subset of $K \times \Gamma_K^d$ is $\Lringtwo$-definable.
\item[$B_d$:] Any $\Lm_2$-definable function $f: Y \subseteq K\times\Gamma_K^d \to \Gamma_K$ is $\Lringtwo$-definable.
\end{enumerate}
The proof goes by induction on $d\geq 1$. The base case $A_1$ corresponds to the assumption of relative $P$-minimality. To prove $B_1$, let 
$f: Y \subseteq K\times\Gamma_K \to \gamma_K$ be a definable function. Then the preparation theorem for $\Gamma$-cells implies that there exists a finite partitioning of $Y$ in ($\Lring$-definable) sets $Y_i$, such that
\[f_{|Y_i}(x, \gamma) = a_i \left(\frac{\gamma -k_i}{n_i}\right) + h_i(x),\] where $a_i, k_i,n_i$ are constants depending on $Y_i$ and $h_i$ is an $\Lm$-definable function $K \to \Gamma_K$. It follows then from $A_1$ that $h_i$, and hence also $f$ is $\Lring$-definable. 

Now assume that $A_d, B_d$ hold for some $d \geqslant 1$,  and let $X\subseteq K\times \Gamma_K^{d+1}$ be a definable set. By the preparation theorem for $\Gamma$-cells (Proposition \ref{prop:partialcd2}), $X$ can be partitioned into finitely many cells of the form 

\[B= \left\{(x,\gamma,\delta)\in D\times\Gamma_K \left|\begin{array}{l} \alpha(x,\gamma) \square_1 \ \delta \ \square_2 \ \beta(x,\gamma), \\
\delta \equiv k\mod n \end{array}\right\}\right.,\]

where $D$ is a definable subset of $K\times\Gamma_K^d$, $\alpha,\beta$ are definable functions, $k<n$ are positive integers.% and for all $(x,\gamma,\delta)\in B$
%
%
%It is then clear from the induction hypothesis (and an iteration of this procedure for $h$), that any function $f: Y \subseteq K\times\Gamma_K^d$
%
%
%Suppose every definable set contained in $K\times\Gamma_K^d$ is $\Lring,2$-definable. As a first step in the induction, consider a definable function
%$f: Y \subseteq K\times\Gamma_K^d \to \gamma_K$. Then the preparation theorem for $\Gamma$-cells implies that there exists a finite partitioning of $Y$ in sets $Y_i$, such that
%\[f_{|Y_i}(x, \gamma_1, \ldots, \gamma_d) = a \left(\frac{\delta -k}{n}\right) + h(x,\gamma_1, \ldots, \gamma_{d-1}),\] where $a, k,n$ are constants depending on $Y_i$ and $h$ a definable function $K \times \Gamma_K^{d}$. It is then clear from the induction hypothesis (and an iteration of this procedure for $h$), that any function $f: Y \subseteq K\times\Gamma_K^d$
%
% and let $X\subseteq K\times \Gamma_K^{d+1}$ be a definable set. 
%
%
%By the preparation theorem for $\Gamma$-cells (Proposition \ref{prop:partialcd2}) $X$ decomposes into finitely many cells of the form 
%
%\[B= \left\{(x,\gamma,\delta)\in D\times\Gamma_K \left|\begin{array}{l} \alpha(x,\gamma) \square_1 \ \delta \ \square_2 \ \beta(x,\gamma), \\
%\delta \equiv k\mod n \end{array}\right\}\right.,\]
%
%where $D$ is a definable subset of $K\times\Gamma_K^d$, $\alpha,\beta$ are definable functions, $k<n$ are positive integers and for all $(x,\gamma,\delta)\in B$
%\[f(x,\gamma,\delta) = a \left(\frac{\delta -k}{n}\right) + h(x,\gamma),\]
%for $a$ an integer and $h$ a definable function {\color{red} What is the relevance of this $f$?}. 

By the induction hypotheses $A_d$ and $B_d$, $D$, $\alpha$, and $\beta$ are $\Lm_{ring,2}$-definable. Therefore, every such cell is $\Lm_{ring,2}$-definable, and hence $A_{d+1}$ holds. The proof of $B_{d+1}$ uses $A_{d+1}$ and exactly the same argument as for $B_1$. 
\end{proof}

Another apparent difference between the concepts of relative $P$-minimality and $\Lringtwo$-minimality is that the later allows languages expanding not only the field structure but also the valued group structure. This seems more general than relative $P$-minimality given that in this notion the Presburger language is fixed on $\Gamma_K$. We show that in fact a $\Lringtwo$-minimal structure $(K,\Gamma_K)$ does not add any more structure to $\Gamma_K$, that the definable sets on $\Gamma_K$ are Presburger definable sets. 

\begin{cor} Let $(K,\Gamma_K)$ be a $\Lringtwo$-minimal structure. Then every definable subset of $\Gamma_K$ (in any power) is $\Lm_{Pres}$-definable.   
\end{cor}
\begin{proof}
Let $X\subseteq \Gamma_K^d$ be a definable subset. By $\Lringtwo$-minimality, $X$ is $\Lringtwo$-definable and by quantifier elimination in $\Lringtwo$, it is easy to see it is $\Lm_{Pres}$-definable.
\end{proof}



We finish by showing that $p$-adically closed fields are relative $P$-minimal in $\Lm_{an}$. Recall that the subanalytic language $\Lm_{an}$ on $K$ is Macintyre's language enriched with the field inverse $^{-1}$ on $K$ extended by $0^{-1}=0$ and, for each convergent power series $f: \mathcal{O}_K^n\to K$, a function symbol for the restricted analytic function
\begin{equation}
x\in K^n\mapsto=
\begin{cases}
f(x) & if x\in \mathcal{O}_K^n\\
0 & \text{otherwise}
\end{cases}
\label{eq:subanalytic}
\end{equation}

We show as stated previously that the structure of subanalytic sets is relative $P$-minimal:

\begin{lem}\label{lem:suban}
Let $(K, \Lm_{\text{an}})$ be the structure of subanalytic sets. The induced two-sorted $\Lm_{\text{an},2}$-structure $(K, \Gamma_K)$ is relatively $P$-minimal. 
\end{lem}
\begin{proof}
Recall that this structure has elimination of quantifiers. Let $X \subseteq K \times \Gamma_K$ be a subanalytic set, defined by a formula $\phi(x,\gamma)$. This formula can then be written in the form
\[\phi(x,\gamma):= \phi_1(x) \wedge (\ord f_1(x), \ldots, \ord f_r(x), \gamma) \in P,\]
where $P \subseteq \Gamma_K^{r}$ is a Presburger set. 
By $P$-minimality, we may assume that there exists an $\Lring$-formula $\psi_1(x)$, such that $K \models \psi_1(x) \leftrightarrow \phi_1(x)$. Moreover, by the preparation theorem for subanalytic functions, we have that $\ord f_i(x) = \frac1{e_i}\ord a_i(x-c)^{\mu_i}$, and hence there exists a semi-algebraic function $h_i(x)$ such that $\ord f_i(x) = \ord h_i(x)$. It is then clear that the (extended) structure of subanalytic sets is relatively $P$-minimal.
\end{proof}

We do not know if the notion of relative $P$-minimality is stronger than classical $P$-minimality. 


%
%
%It is then clear from the induction hypothesis (and an iteration of this procedure for $h$), that any function $f: Y \subseteq K\times\Gamma_K^d$
%
%
%Suppose every definable set contained in $K\times\Gamma_K^d$ is $\Lring,2$-definable. As a first step in the induction, consider a definable function
%$f: Y \subseteq K\times\Gamma_K^d \to \gamma_K$. Then the preparation theorem for $\Gamma$-cells implies that there exists a finite partitioning of $Y$ in sets $Y_i$, such that
%\[f_{|Y_i}(x, \gamma_1, \ldots, \gamma_d) = a \left(\frac{\delta -k}{n}\right) + h(x,\gamma_1, \ldots, \gamma_{d-1}),\] where $a, k,n$ are constants depending on $Y_i$ and $h$ a definable function $K \times \Gamma_K^{d}$. It is then clear from the induction hypothesis (and an iteration of this procedure for $h$), that any function $f: Y \subseteq K\times\Gamma_K^d$
%
% and let $X\subseteq K\times \Gamma_K^{d+1}$ be a definable set. 
%
%
%By the preparation theorem for $\Gamma$-cells (Proposition \ref{prop:partialcd2}) $X$ decomposes into finitely many cells of the form 
%
%\[B= \left\{(x,\gamma,\delta)\in D\times\Gamma_K \left|\begin{array}{l} \alpha(x,\gamma) \square_1 \ \delta \ \square_2 \ \beta(x,\gamma), \\
%\delta \equiv k\mod n \end{array}\right\}\right.,\]
%
%where $D$ is a definable subset of $K\times\Gamma_K^d$, $\alpha,\beta$ are definable functions, $k<n$ are positive integers and for all $(x,\gamma,\delta)\in B$
%\[f(x,\gamma,\delta) = a \left(\frac{\delta -k}{n}\right) + h(x,\gamma),\]
%for $a$ an integer and $h$ a definable function {\color{red} What is the relevance of this $f$?}. 


%If analogues for cell decompositon and preparation theorems were proven for $P$-minimal structures, both notions would be equivalent. We leave this question open. 




