\documentclass[11pt]{amsart}
%\documentclass{amsart}
\usepackage{amssymb, amsthm}
\usepackage{amsfonts}
\usepackage{amscd}
\usepackage[T1]{fontenc}
%\usepackage[all]{xypic}
%\usepackage{amsmath}
\usepackage{color,mdwlist}
%\usepackage{showkeys}
%\usepackage{tikz}
%\usepackage{amsrefs}

%\usepackage{hyperref}

% replaced \newcommand by \long\def to allow paragraph breaks inside.
\long\def\comment#1{\marginpar{{\footnotesize\color{red} #1\par}}}
\long\def\commentimmi#1{\marginpar{{\footnotesize\color{green} #1\par}}}
\long\def\change#1{{\color{blue} #1}}
\newcommand\green[1]{{\color{green} #1}}
% Strike things out: (doesn't work well with line breaks)
\newbox\removebox
\newcommand\remove[1]{%
\setbox\removebox=\ifmmode\hbox{$#1$}\else\hbox{#1}\fi%
\leavevmode
\rlap{\textcolor{blue}{\vrule height0.8ex depth-0.6ex width\wd\removebox}}%
\box\removebox
}
\long\def\bigremove#1{%
\par\setbox\removebox=\vbox{#1}%
\vbox{%
\vbox to0pt{\hbox{\tikz\draw[color=blue,thick] (0,0) -- (\wd\removebox,-\ht\removebox)  (\wd\removebox,0) -- (0,-\ht\removebox);}}
\box\removebox
}
}

\long\def\jc#1{{\color{green} #1}}


\usepackage{mathrsfs} %for the \mathscr


\newcommand{\Cexp}{\mathrm{C}^{\mathrm{exp}}}
\newcommand{\ICexp}{\mathrm{IC}^{\mathrm{exp}}}
\newcommand{\cCexp}{\cC^{\mathrm{exp}}}
\newcommand{\cQexp}{\cQ^{\mathrm{exp}}}
\newcommand{\IcCexp}{\mathrm{I}\cC^{\mathrm{exp}}}
\newcommand{\AO}[1][]{\cA_{\cO#1}}
\newcommand{\BO}[1][]{\cB_{\cO#1}}
\newcommand{\CO}[1][]{\mathcal{C}_{\cO#1}}
\newcommand{\Vol}{\operatorname{Vol}}
\newcommand{\Int}{\operatorname{Int}}
\newcommand{\Bdd}{\operatorname{Bdd}}
\newcommand{\Iva}{\operatorname{Iva}}
\newcommand{\Lm}{\mathcal{L}}
\newcommand{\Lring}{\Lm_{\text{ring}}}



\def\RDefe{Q^{\rm exp}}


\newcommand{\iso}{\mathrel{\overset{\sim}{\smash{\longrightarrow}\vrule height.3ex width0ex\relax}}}


\def\deg{\operatorname{deg}}
\def\ac{{\overline{\rm ac}}}
\def\Supp{\operatorname{Supp}}
\def\limind{\mathop{\oalign{lim\cr
\hidewidth$\longrightarrow$\hidewidth\cr}}}
\def\Lp{{{\mathbf L}_{\rm PR}}}
\def\Eu{\operatorname{Eu}}
\def\Var{\operatorname{Var}}
\def\Def{\operatorname{Def}}
\def\Cons{\operatorname{Cons}}
\def\RDef{\operatorname{RDef}}
\def\GDef{\operatorname{GDef}}
\def\tame{\operatorname{tame}}
\def\SA{\operatorname{SA}}
\def\Leq{(\cL_{K})^{\mathrm{eq}}}
\def\LPas{\cL_{\rm DP}}
\def\LPre{\cL_{\rm DP,P}}
\def\LPres{\cL_{\rm Pres}}
\def\Ord{\operatorname{Ord}}
\def\DefR{\operatorname{Def}_{R}}
\def\GDefR{\operatorname{GDef}_{R}}
\def\Kdim{\operatorname{Kdim}}
\def\Pres{\operatorname{Pres}}
\def\CMC{\operatorname{CMC}}
\def\ICMC{\operatorname{ICMC}}
\def\pre{\operatorname{pre}}
\def\kvar{{K_0 (\operatorname{Var}_k)}}
\def\VarR{\Var_{R}}
\def\GL{\operatorname{GL}}
\def\Gr{\operatorname{Gr}}
\def\SL{\operatorname{SL}}
\def\eff{\operatorname{eff}}
\def\res{\operatorname{res}}
\def\rv{\operatorname{rv}}
\def\RV{\operatorname{RV}}

\def\longhookrightarrow{\mathrel\lhook\joinrel\longrightarrow}
\let\cal\mathcal
\let\got\mathfrak
\def\gP{{\got P}}
\def\ccF{{\mathscr F}}
\def\pp{{\mathbb p}}
\def\11{{\mathbf 1}}
\def\AA{{\mathbb A}}
\def\BB{{\mathbb B}}
\def\CC{{\mathbb C}}
\def\DD{{\mathbb D}}
\def\EE{{\mathbb E}}
\def\FF{{\mathbb F}}
\def\GG{{\mathbb G}}
\def\HH{{\mathbb H}}
\def\II{{\mathbb I}}
\def\JJ{{\mathbb J}}
\def\KK{{\mathbb K}}
\def\LL{{\mathbb L}}
\def\MM{{\mathbb M}}
\def\NN{{\mathbb N}}
\def\N{{\NN}}
\def\OO{{\mathbb O}}
\def\PP{{\mathbb P}}
\def\QQ{{\mathbb Q}}
\def\Q{{\QQ}}
\def\RR{{\mathbb R}}
\def\R{{\RR}}
\def\SS{{\mathbb S}}
\def\TT{{\mathbb T}}
\def\UU{{\mathbb U}}
\def\VV{{\mathbb V}}
\def\WW{{\mathbb W}}
\def\XX{{\mathbb X}}
\def\YY{{\mathbb Y}}
\def\ZZ{{\mathbb Z}}
\def\Z{{\ZZ}}

\def\gM{{\mathfrak M}}
\def\cA{{\mathcal A}}
\def\cB{{\mathcal B}}
\def\cC{{\mathscr C}}
\def\cD{{\mathcal D}}
\def\cE{{\mathcal E}}
\def\cF{{\mathcal F}}
\def\cG{{\mathcal G}}
\def\cH{{\mathcal H}}
\def\cI{{\mathcal I}}
\def\cJ{{\mathcal J}}
\def\cK{{\mathcal K}}
\def\cL{{\mathcal L}}
\def\cM{{\mathcal M}}
\def\cN{{\mathcal N}}
\def\cO{{\mathcal O}}
\def\cP{{\mathcal P}}
\def\cQ{{\mathcal Q}}
\def\cR{{\mathcal R}}
\def\cS{{\mathcal S}}
\def\cT{{\mathcal T}}
\def\cU{{\mathcal U}}
\def\cV{{\mathcal V}}
\def\cW{{\mathcal W}}
\def\cX{{\mathcal X}}
\def\cY{{\mathcal Y}}
\def\cZ{{\mathcal Z}}
\def\llp{\mathopen{(\!(}}
\def\llb{\mathopen{[\![}}
\def\rrp{\mathopen{)\!)}}
\def\rrb{\mathopen{]\!]}}


\newtheorem{thm}[subsection]{Theorem}
\newtheorem*{thm*}{Theorem}
\newtheorem{lem}[subsection]{Lemma}
\newtheorem{cor}[subsection]{Corollary}
\newtheorem{prop}[subsection]{Proposition}
\newtheorem{conj}[subsection]{Conjecture}
\newtheorem{problem}[subsection]{Problem}
\newtheorem{claim}[subsection]{Claim}
\newtheorem{maintheorem}[subsection]{Main Theorem}
\newtheorem{def-prop}[subsection]{Proposition-Definition}
\newtheorem{def-theorem}[subsection]{Theorem-Definition}
\newtheorem{def-lem}[subsection]{Lemma-Definition}


\theoremstyle{definition}
\newtheorem{defn}[subsection]{Definition}
\newtheorem{example}[subsection]{Example}
\newtheorem{xca}[subsection]{Exercise}



\theoremstyle{remark}
\newtheorem{remark}[subsubsection]{Remark}
\newtheorem{remarks}[subsubsection]{Remarks}
\newtheorem{remark - ques}[subsubsection]{Remark/Quesion}


\theoremstyle{plain}


\newcommand{\todo}[1]{\marginpar{{\scriptsize #1\par}}}

\numberwithin{equation}{subsection}

%\newcommand{\sur}[2]{\genfrac{}{}{0pt}{}{#1}{#2}}  use \binom
\renewcommand{\theequation}{\thesubsection.\arabic{equation}}

\newenvironment{subcase}[1]%
  {\par\medskip\noindent #1\par\begingroup%
    \advance\leftskip by 1em\advance\rightskip by 1em}%
  {\par\endgroup}

\def\GM{{\GG_{\mathrm{m}}}}

\DeclareMathOperator*{\lcm}{lcm}
\DeclareMathOperator*{\Spec}{Spec}
\DeclareMathOperator*{\Specf}{Spf}
\DeclareMathOperator*{\Spf}{Spf}
\DeclareMathOperator*{\sqc}{\sqcup}
%\DeclareMathOperator*{\sq}{\square}
\newcommand{\sq}{\mathrel{\square}}
\newcommand{\ord}{\operatorname{ord}}
\newcommand{\Jac}{\operatorname{Jac}}
%Absolute value notation
\newcommand{\abs}[1]{\lvert#1\rvert}

%\newcommand{\statement}[1]{\begin{itemize}\item #1\end{itemize}}
\newcommand{\statement}[1]{\\\hspace*{\fill}#1\hspace*{\fill}\\}


%\newcommand{\GL}{\operatorname{GL}}


%    Blank box placeholder for figures (to avoid requiring any
%    partular graphics capabilities for printing this document).
\newcommand{\blankbox}[2]{%
  \parbox{\columnwidth}{\centering
%    Set fboxsep to 0 so that the actual size of the box will match the
%    given measurements more closely.
    \setlength{\fboxsep}{0pt}%
    \fbox{\raisebox{0pt}[#2]{\hspace{#1}}}%
  }%
}

\parindent=0pt
\begin{document}

\setcounter{tocdepth}{1} % Show subsection in table of contents


\title%[Transfer principles for integrability and boundedness]
{Integration in P-minimal structures -- research program  }

\begin{abstract}
\end{abstract}

%\thanks{The research leading to these results has received funding from the European Research Council under the European Community's Seventh Framework Programme (FP7/2007-2013) / ERC Grant Agreement nr. 246903 NMNAG}

\maketitle

%\tableofcontents

% !TEX root = nov.tex
\section{Introduction}


Let $K$ be a $p$-adic field, $\Gamma_K$ be its value group, $\cO_K$ its valuation ring, $\cM_K$ its maximal ideal and $q_K$ the number of elements of its residue field $k_K$. Write $\pi$ for a uniformizer of $K$. Let $\cL_{\rm ring}$ be the ring language and let $\cL$ be a language containing $\cL_{\rm ring}$ such that $(K,\cL)$ is P-minimal. Associated to $\cL$ is the two-sorted language $\cL_2$ which consists of a sort for $K$ in $\cL$, a sort for a sort for $\ZZ\cup\{+\infty\}$ in the language of Presburger arithmetic $\cL_{Pres}$ and the valuation map $\ord: K\to \ZZ\cup\{+\infty\}$. Denef introduced a certain class of functions, which has been shown to be stable under integration in the semi-algebraic and in the subanalytic case, see \cite{denef-2000}, \cite{Clu-2003}, and \cite{Clu-Gor-Hal-14} for the most convenient dealing with integrability conditions. The analogous class of functions, build up from our general P-minimal structure is not known to have this stability property. Here, we suggest a notion of relative P-minimality, for which we show stability under integration.

\begin{defn}

Call the structure $(K,\cL)$ relatively P-minimal (relative to the sort $\ZZ$), if, for any $(K',\cL)$ which is elementary equivalent to $(K,\cL)$, any $\cL_2(K')$-definable subset $X$ of $K'\times \Gamma_{K'}^\ell$ is already $\cL_{\rm ring, 2}(K)$-definable.

\end{defn}
Here, $\Gamma_{K'}$ stands for the value group of $K'$.

\begin{remark - ques}\label{rq1} Maybe the above definition with $\ell=1$ instead of general $\ell$ is enough, or maybe even equivalent to the version for general $\ell$. 
\end{remark - ques}


\begin{remark}\label{r1} Maybe it is convenient to also require definable skolem functions at some point, but one of the goals is to avoid such condition. 
\end{remark}


\begin{defn}
Let $X$ be an $\cL_2$-definable set. Write $\AA_{q_K}$ for the ring \[\AA_{q_K}:=\ZZ\left[q_K, q_k^{-1}, \left(\frac1{1-q_k^{-i}}\right)_{i\in \NN, i>0}\right].\]
Call a function $f:X\to \QQ$ $\cL$-constructible (or $\cL_2$-constructible) if it is contained in the $\AA_{q_K}$-algebra generated by functions of the forms
\begin{enumerate}
\item $\alpha:X\to \ZZ$
\item $X\to \ZZ:x\mapsto q_K^{\beta(x)}$,
\end{enumerate}
where $\alpha$ and $\beta$ are $\cL_2$-definable and $\ZZ$-valued.  
\end{defn}


When $\cL$ is $\cL_{\rm ring}$ or the subanalytic language on $K$ (or some intermediary languages as in \cite{CLip}), the classes of constructible functions are known to be stable under integration. This is proved using a form of preparation theorem for constructible functions, based on cell decomposition. In P-minimal structures, there are several difficulties related to cell decomposition. Mourgues \cite{mou-09} proved a somewhat weaker form of cell decomposition under the assumption of having definable Skolem functions on $(K,\cL)$, but this form of cell decomposition seems not enough to show stability under integration.

Here we investigate the following.

\begin{thm}\label{thm1}
Suppose that $(K,\cL)$ is relatively P-minimal. Then the class of $\cL$-constructible functions is stable under integration, in the sense of Section 3 of \cite{Clu-Gor-Hal-14}.
\end{thm}

Also rationality results are so far only known for the semi-algebraic setting \cite{denef-84}, the subanalytic setting \cite{denef-vdd-88} (and thus also for any sublanguage).

As a side result we also obtain the following.

\begin{thm}\label{thm:rationality}
Suppose that $(K,\cL)$ is relatively P-minimal. Let $X$ be a $\cL_2$-definable subset of $\cO_K^n\times \NN$, and let $a_n$ be the Haar measure of $X_n:=\{x\in \cO_K^n\mid (x,n)\in X\}$ for each $n\geq 0$. Then the series $\sum_{i\geq 0} a_i T^i$ is rational. 
\end{thm}
Here, we normalize the Haar measure on $K^n$ so that $\cO_K^n$ has measure $1$. 

\

Let $(K,\Gamma_K)$ be a $\cL_2$-structure. For notational purposes we will fix a definable set $S \subseteq K^{m_0} \times \Gamma_K^{m_0}$ which we call a \emph{parameter set}. For a definable set $X\subseteq S\times K$ and $s\in S$, $X_s:=\{x\in K: (s,x)\in X\}$ denotes the fiber over $s$. Analogously, for a definable function $f:X\rightarrow \Gamma_K$, $f_s$ denotes the function $f(s,\cdot):X_s\rightarrow \Gamma_K$. Given two sets $A$ and $B$, we denote by $\Pi_{A}:A\times B\to A$ for the projection onto $A$ and by $\Pi_B:A\times B\to B$ the projection onto $B$. For a positive integer $n\geq 1$, $A^{\leq n}$ denotes $\bigcup_{i=1}^n A^i$. We start defining cells in this context: 


\begin{defn}[Cells] Let $(K,\Gamma_K)$ be a $\cL_2$-structure.  
\begin{enumerate}
\item A subset $C\subseteq S\times K$ is a $K$-cell if it is of the form 
\[C = \left\{(s,t) \in D\times K \ \left| \ \begin{array}{l} \alpha(s) \square_1 \ \ord(t-c(s)) \ \square_2 \ \beta(s),\\ \ord(t-c(s)) \equiv k\mod n,\\ \ac_{m}(t-c(s)) = \xi \end{array} \right\}\right.,\]
where $D$ is a $\cL_2$-definable subset of $S$, $c$ is a function $c:D\to K$, $\alpha, \beta$ are $\Lm_2$-definable functions $D\to\Gamma_k$, $k, n, m \in \NN$, $\xi \in \ac_{m}(K)$ and the squares $\square_i$ may denote $<$ or \emph{no condition}. If the function $c$ is $\cL_2$-definable, we say that the cell has \emph{definable centers}.  
\item A subset $B\subseteq S\times \Gamma_K$ is a $\Gamma$-cell if it is of the form
\[B= \left\{(s,\gamma)\in D\times \Gamma_K \left|\begin{array}{l} \alpha(s) \square_1 \ \gamma \ \square_2 \ \beta(s), \\
\gamma \equiv k\mod n \end{array}\right\}\right.,\]
where $D$ is an $\cL_2$-definable subset of $S$, $\alpha, \beta$ are $\Lm_2$-definable functions $D\to\Gamma_k$, $k, n\in \NN$ and again the squares $\square_i$ may denote $<$ or \emph{no condition}.  
 langauge. 
\end{enumerate}
\end{defn}

We call \emph{cell} a subset which is either a $K$-cell or a $\Gamma$-cell. Using this terminology, a semi-algebraic cell is a cell with respect to the language $\cL_{ring,2}$ and a sub-analytic cell is a cell with respect to $\cL_{an,2}$, where $\cL_{an}$ is the sub-analytic language on $K$, i.e., Macintyre's language enriched with the field inverse $^{-1}$ on $K$ extended by $0^{-1}=0$ and, for each convergent power series $f: \cO^n
\to K$, a function symbol for the
restricted analytic function
\[x \mapsto 
\begin{cases}
f(x) &\text{ if }x \in \cO^n\\
0 &\text{ otherwise }.
\end{cases}\]
We will fix some notation and terminology regarding cells that will be used through the rest of the article. Let $\square = (\square_1, \square_2) \in \{\emptyset, <\}^{2}$ (we treat $\emptyset$ as ``no-condition'') $(k,n,m)$ be a triple of positive integers such that $k<n$ and $\xi\in \ac_{m}(K)$. Let $P_K$ denote the set of all tuples $(\square,k,n,m,\xi)$ and $P_\Gamma$ be the set of all tuples $(\square,k,n)$. For $\delta=(\square,k,n,m,\xi)\in P_K$, $\lambda=(\square,k,n)\in P_{\Gamma}$, $\alpha,\beta,\gamma$ variables of type $\Gamma$ and $x,y$ variables of type $K$, we define the formulas
\iffalse
\begin{align}\label{defcells}
\begin{split}
C_{1,\delta}(x,y,\alpha,\beta)&:= (\alpha\square_{1}\ \ord(x -y)\\square_{2}\ \beta)\\
C_{2,\delta}(x,y)&:= ({\ord(x-y)\equiv k \ \text{mod } n})\\
C_{3,\delta}(x,y)&:= (\ac_{m}(x-y)=\xi)\\
C_\delta(x,y,\alpha,\beta)&:=C_{1,\delta}(x,y,\alpha,\beta)\wedge C_{2,\delta}(x,y)\wedge C_{3,\delta}(x,y)\\
\\
B_{1,\lambda}(\alpha,\beta,\gamma)&:= (\alpha\square_{1} \gamma \ \square_{2}\ \beta)\\
B_{2,\lambda}(\gamma)&:= ({\gamma\equiv k \ \text{mod } n})\\
B_{\lambda}(\alpha,\beta,\gamma)&:=B_{1,\lambda}(\alpha,\beta,\gamma)\wedge B_{\lambda}(\gamma).
\end{split}
\end{align}
\fi

\begin{equation}\label{defKcells}
C_\delta(x,y,\alpha,\beta) = \left(\begin{array}{l} \alpha \square_1 \ \ord(x-y) \ \square_2 \ \beta \ \wedge\\ \ord(x-y) \equiv k\mod n \ \wedge\\ \ac_{m}(x-y) = \xi \end{array} \right),
\end{equation}
\begin{equation}\label{defGammacells}
B_\lambda(\alpha,\beta,\gamma):= \left(\begin{array}{l} \alpha \square_1 \ \gamma \ \square_2 \ \beta \ \wedge \\
\gamma \equiv k\mod n \end{array}\right).
\end{equation}

Notice that for a definable subset $C\subseteq S\times K$, we can now express the fact that $C$ is a cell with definable centers claiming the existence of definable functions $\alpha,\beta:\Pi_S(C)\to\Gamma_K$, $c:\Pi_S(C)\to K$ and of a tuple $\delta\in P_K$ such that 
\[C = \left\{(s,t) \in \Pi_S(C)\times K \ \left| \ (K,\Gamma_K)\models C_\delta(t,c(s),\alpha(s),\beta(s))\right\}\right..
\]
Analogously, for a definable subset $B\subseteq S\times \Gamma_K$, $B$ is a cell if there are definable functions $\alpha,\beta:\Pi_S(C)\to\Gamma_K$ and an tuple $\lambda\in P_\Gamma$ such that  
\[B = \left\{(s,\gamma) \in \Pi_S(B) \times \Gamma \ \left| \ (K,\Gamma_K)\models B_\lambda(\alpha(s),\beta(s),\gamma)\right\}\right..
\]

\subsection{$P$-minimality versus relative $P$-minimality}
How does our notion of relative $P$-minimality relate to the classical notion, as introduced by Haskell and Macpherson \cite{has-mac-97}?

When working in a $p$-adic setting, it is quite natural to distinguish the value group as a separate sort. Cluckers \cite{clu-presb03} made the following observations in this context:
\begin{thm}\label{thm:semialgpres}
Let $(K, \Lm)$ be a $P$-minimal field with $P$-minimal theory. 
\item For any $\Lm$-definable set $X \subseteq (K^{\times})^m$, the set
\[\ord(X):=\{(\ord x_1, \ldots, \ord x_m) \mid (x_1, \ldots, x_m) \in X\}\]
is $\Lm_{\text{Pres}}$-definable.
\item Let $S \subseteq \Gamma_K^m$ be a Presburger-definable set. Then the set
\[\{\ord^{-1}(S):= \{ (x_1, \ldots, x_m) \in X \mid \ord x \in S\}\]
is $\Lring$-definable.
\end{thm} 
Hence, the Presburger language on the value group sort is induced by $\Lring$ on the field sort, and the assumption of $P$-minimality ensures that this is sufficient even for bigger $\Lm$. Hence, it is quite natural to also impose minimality conditions on the value group sort. For example, the structure of subanalytic sets will satisfy this new minimality notion:

\begin{lem}
Let $(K, \Lm_{\text{an}})$ be the structure of subanalytic sets. The induced two-sorted structure $(K, \Gamma_K, \Lm_{\text{an},2})$ is relatively $P$-minimal. 
\end{lem}
\begin{proof}
Recall that this structure has elimination of quantifiers. Let $X \subseteq K \times \Gamma_K^{\ell}$ be a subanalytic set, defined by a formula $\phi(x,\gamma)$. This formula can then be written in the form
\[\phi(x,\gamma):= \phi_1(x) \wedge (\ord f_1(x), \ldots, \ord f_r(x), \gamma) \in P,\]
where $P \subseteq \Gamma_K^{r+ \ell}$ is a Presburger set. 
By $P$-minimality, we may assume that there exists an $\Lring$-formula $\psi_1(x)$, such that $K \models \psi_1(x) \leftrightarrow \phi_1(x)$. Moreover, by the preparation theorem for subanalytic functions, we have that $\ord f_i(x) = \frac1{e_i}\ord a_i(x-c)^{\mu_i}$, and hence there exists a semi-algebraic function $h_i(x)$ such that $\ord f_i(x) = \ord h_i(x)$. It is then clear that the (extended) structure of subanalytic sets is relatively $P$-minimal.
\end{proof}
In general, the notion of relative $P$-minimality may wel be stronger than classical $P$-minimality. {\color{red} Might this be true: both notions are equivalent if $P$-minimality would have the nice properties we want it to have, i.e. cell decompositon and function preparation}
% !TEX root = nov.tex
\section{A preparation theorem for relative P-minimality}

The following is a version of the well-known cell decomposition for semi-algebraic sets. 

\begin{prop}[Semi-algebraic cell decomposition, Theorem 3.3.2 in \cite{Clu-Gor-Hal-14}]\label{prop.celldecomp}
Let $X \subseteq  S \times K$ and $f_j:X \to\Gamma_K $ be $L_{ring,2}$-definable functions for $j=1,\ldots,r$. Then there exists a finite partition of $X$ into $p$-adic cells $C_i$ 
\[C_i = \left\{(s,t) \in D\times K \ \left| \ \begin{array}{l} \alpha_i(s) \square_1 \ \ord(t-c_i(s)) \ \square_2 \ \beta_i(s),\\ \ord(t-c_i(s)) \equiv k_i\mod n_i,\\ \ac_{m_i}(t-c_i(s)) = \xi_i \end{array} \right\}\right.,\]
such that for each occurring cell $C_i$ one has that for each $(s,t)\in C_i$
\[f_j(s,t)=  h_{ij}(s)+ a_{ij}\frac{\ord(t- c_i(s))- k_i}{n_i},
\]
for $a_{ij}$ integers and $h_{ij}: D\to \Gamma_K$ $L_{ring,2}$-definable functions for $j=1,\ldots,r$. 
\end{prop}

An analogous statement is true where semi-algebraic is replaced by sub-analytic. Notice that in these two cases all $K$-cells have definable centers. 

\begin{lem}\label{lem.semialgfnc}
Let $f: X \subseteq K \to \Gamma_K$ be an $\Lm_2$-definable function. Then, there exists an $\cL_{Ring,2}$-definable function {$h: X\to K$} such that \[f(x) = \ord h(x)\] for all $x \in X$. 
\end{lem}
\begin{proof}

It follows from our assumptions that $\mathrm{Graph}(f)$ is an $\Lm_{\text{ring}, 2}$-definable subset of $K \times \Gamma_K$. Now consider the related set
\[ G(f) := \{(x,y) \in K^2 \mid (x, \ord y) \in \mathrm{Graph}(f)\}.\]
Clearly, G(f) is semi-algebraic, that is, $\cL_{ring}$-definable. Since the structure of semi-algebraic sets admits definable Skolem functions by a result of van den Dries \cite{vdd-84}, one can find an $\Lring$-definable function $h: \Pi_x(G(f)) \to K$, such that $(x, h(x)) \in G(f)$. 
\end{proof}

\begin{thm}[Preparation theorem for relative P-minimality] 
\label{thm:partialprep} 
Let $X \subseteq S \times K$ and $f:X\rightarrow \Gamma_k$ be an $\Lm_2$-definable function. There exists a finite decomposition of $X$ into cells $C$ 
\[C = \left\{(s,t) \in D\times K \ \left| \ \begin{array}{l} \alpha(s) \square_1 \ \ord(t-c(s)) \ \square_2 \ \beta(s),\\ \ord(t-c(s)) \equiv k\mod n,\\ \ac_{m}(t-c(s)) = \xi \end{array} \right\}\right..\]
On each such cell $C$, there is a $\Lm_2$-definable function $\gamma:D\to\Gamma_K$ and an integer $a\in \ZZ$ such that for all $(s,t) \in C$, 
\[f(s,t) = \gamma(s) + a\frac{\ord(t-c(s))-k}{n}.\]
We stress that the functions $c: S\to K$ are NOT necessarily $\Lm_2$-definable.
\end{thm}
\begin{proof}

Fix a parameter $s \in \Pi_S(X)$. By Lemma \ref{lem.semialgfnc} there is a $\Lring$-definable function {$h_s: X_s\to K$} such that $f_s(t) = \ord h_s(t)$ for all $t \in X_s$ (we do not assume that $h_s$ is uniformly definable for all $s\in \Pi_S(X)$). By Proposition \ref{prop.celldecomp}, there exists a finite partition of $X_s$ into cells of the form 
\begin{equation}\label{cell1}
C_s = \left\{t \in K \ \left| \ \begin{array}{l} \alpha_s \square_1 \ \ord(t-c(s)) \ \square_2 \ \beta_s,\\ \ord(t-c_s) \equiv k_s\mod n_s,\\ \ac_{m_s}(t-c_s) = \xi_{m_s,s} \end{array} \right\}\right.,
\end{equation}
where $\alpha_s, \beta_s \in \Gamma_k, c_s \in K$, $(\square,k_s, n_s, m_s,\xi_{m_s,s})\in P_K$ and on each cell $C_s$ 
\begin{equation}\label{preparation}
f_s(t)=\ord h_s(t) =\gamma_s + a_s\frac{\ord(t-c_s)-k_s}{n_s},
\end{equation}
for some $a_s\in \ZZ$ and $\gamma_s\in \Gamma_K$.

\

Given a cell $C_s$ as in \ref{cell1}, the tuple $(\square,k_s,n_s,m_s,\xi_{m_s,s})$ is called the \emph{type} of the cell. Notice that for $s\in S$, the decomposition of $X_s$ might contain different cells of the same type. 

\begin{claim} 
There is a natural number $N\geq 1$ such that for every $s\in S$, there is a cell decomposition of $X_s$ such that each cell satisfies \ref{eq:1}, $n_s,m_s,|a_s|<N$ and moreover there occur at most $N$ cells of the same type.
\end{claim}

\begin{proof}[Proof of the claim:]

The claim will follow by a standard compactness argument. For a positive integer $N$, let 
\begin{align}\label{defPK}
\begin{split}
P_{K,N}&:= \{(\square, k,n,m,\xi)\in P_K \mid n<N, m<N\}\\
E_{K,N} &:= \bigsqcup_{i=1}^N P_{K,N},
\end{split}
\end{align}
the disjoint union of $N$ copies of $P_{K,N}$. Notice that $E_{K,N}$ is a finite set. For $\delta=(\square, k,n,m,\xi)\in P_{K,N}$ we let $k_{\delta}:=k$, $n_\delta:=n$, and so forth. Given $J\subseteq E_{K,N}$, $\{\delta_1,\ldots,\delta_{|J|}\}$ a fixed enumeration of $J$, $y\in K^{|J|}$ and $\alpha=(\alpha_1,\alpha_2)\in \Gamma_K^{2|J|}$, we put 
\[C_J(y,\alpha)=\bigcup_{i=1}^{|J|} C_{\delta_i}(K,y_i,\alpha_{1i},\alpha_{2i}).
\]
Fix $J\subseteq E_{K,N}$ and $\{\delta_1,\ldots,\delta_{|J|}\}$ an enumeration of $J$, and let $a\in\ZZ^{|J|}$ be such that such that $|a_i|<N$ for all $1\leq i\leq |J|$. Let $t,x,y$ be variables of type $K$, $t$ of length 1, $x$ of the same length as elements in $S$ and $y$ of length $|J|$. Let $\alpha=(\alpha_1,\alpha_2,\alpha_3)$ be variables of type $\Gamma_K$, all $\alpha_1,\alpha_2$ and $\alpha_3$ of length $|J|$. We define the formula $\phi_{J,a}(x,y,\alpha)$
by

\small
\begin{equation}\label{cellformula}
\left(\begin{array}{l}
X_x=C_J(y,(\alpha_1,\alpha_2))\ \wedge \\
\displaystyle\bigwedge_{1\leq i\leq |J|} \forall t \left( C_{\delta_i}(t,y_i,\alpha_{1i},\alpha_{2i})\to \left(f_x(t)=\alpha_{3i}+a_i\frac{\ord(t-y_i))-k_{\delta_i}}{n_{\delta_i}}\right)\right)\end{array}\right).\\
\end{equation}
Roughly, this formula states that the set $X_x$ can be decomposed into finitely many cells satifying condition \ref{preparation}. Define the set $W_N$ by 
\begin{equation}\label{WN}
W_N:=\{(J,a): J\subseteq E_{K,N}, a\in \ZZ^{|J|}, |a_i|<N \text{ for all } 1\leq i\leq |J|\},
\end{equation}
and consider the set of formulas 
\[\Sigma(x):=\left\{ \bigwedge_{(J,a)\in W_N} \neg (\exists y\in K^{|J|}\exists \alpha\in \Gamma_{K}^{3|J|}\phi_{J,a}(x,y,\alpha)) \mid N\in\mathbb{N}^*\right\}.
\]

By cell decomposition, $\Sigma(x)$ is inconsistent. Thus, by compactness, there exists a finite subset $\Sigma_0(x)$ which is inconsistent. Note that for every finite subset $\Sigma_0(x)\subseteq\Sigma(x)$, one can find a positive integer $N$ such that 
\[\bigwedge_{(J,a)\in W_N} \neg (\exists y\in K^{|J|}\exists \alpha\in \Gamma_{K}^{3|J|}\phi_{J,a}(x,y,\alpha)) \models \Sigma_0(x).
\end{equation*}

This implies that there must exist $N>0$ such that for every $s\in \Pi_S(X)$  
\[(K,\Gamma_K)\models \bigvee_{(J,a)\in W_N} \exists y\in K^{|J|}\exists \alpha\in \Gamma_{K}^{3|J|}\phi_{J,a}(s,y,\alpha)) 
\]
which is exactly what we wanted. 
\end{proof}

Let $N$ be given by the claim and $W_N$ as defined in \ref{WN}. Since $W_N$ is finite, one can put a total ordering $\lessdot$ on it. We will also put an alternative ordering $\lhd$ on the value group $\Gamma_K$ defined by :
\[x \lhd y \Leftrightarrow
(0 \leqslant x <y) \vee (0< x \leqslant -y) \vee (0< -x < y) \vee (0< -x < -y).\]
This produces a total ordering on $\Gamma_K$ which can be extended to $\Gamma_K^k$ lexicographically. We also denote this extension by $\lhd$. The important property of the order $\lhd$ is that every $\Lm_2$-definable set of $\Gamma_K$ has a $\lhd$-smallest element (for a proof of this, see the appendix, in particular \ref{preswellorder}).

\

Now consider the map \[\sigma: S \to W_N \times (\Gamma_K)^{\leq 3|W_N|}: s \mapsto (\sigma_1(s), \sigma_2(s)),\]where $\sigma_1, \sigma_2$ are defined as follows:
\begin{itemize}
\item put $\sigma_1(s)= (J,a)$, if $(J,a)$ is the $\lessdot$-smallest element of $W_N$ such that  
$$(K,\Gamma_K)\models \exists y\in K^{|J|}\exists \alpha\in \Gamma_{K}^{3|J|}\phi_{J,a}(s,y,\alpha).$$
The claim ensures the existence of at least one such $(J,a)$ in $W_N$.    
\item let $\phi_2(s)$ be the $\lhd$-smallest tuple $\alpha\in \Gamma_K^{3|J|}$ such that 
$$(K,\Gamma_K)\models \exists y\in K^{|J|}\phi_{\sigma_1(s)}(s,y,\alpha).$$
\end{itemize}
It is clear that the function $\sigma$ will be $\Lm_2$-definable, once we fix a representation for the finite index set $W_N$. From this information one can then reconstruct the cell decomposition for $X$ and the linear functions satisfying \ref{eq:1}, in the following way. Let $S_0:=\Pi_S(X)$ and for $\lambda\in \sigma_1(S_0)$ let $S_{\lambda}$ be the set $\{s \in S_0 \mid \sigma_1(s)=\lambda\}$. For $\lambda=(J,a)\in \sigma_1(S_0)$, $1\leq j\leq |J|$ and $i=1,2,3$ let $\alpha_{\lambda ji}: S_\lambda\to \Gamma_K$ be defined by 

\[\sigma_2(s)=\left(
\begin{array}{l}
\alpha_{\lambda 11}(s),\ldots,\alpha_{\lambda j1}(s),\ldots, \alpha_{\lambda |J|1}(s),\\
\alpha_{\lambda 12}(s),\ldots,\alpha_{\lambda j2}(s),\ldots, \alpha_{\lambda |J|2}(s),\\
\alpha_{\lambda 13}(s),\ldots,\alpha_{\lambda j3}(s),\ldots, \alpha_{\lambda |J|3}(s),
\end{array}\right).\] 

This produces a finite partition of $X$ into cells

\[X = \bigcup_{\lambda=(J,a)\in \sigma_1(S_0)}\bigcup_{1\leq j\leq |J|}\left( \prod_{s\in S_\lambda} \{s\}\times C_{\lambda}(K,c(s),\alpha_{\lambda 1j}(s),\alpha_{\lambda 2j}(s))\right)\\
\]

where $c:\Pi_S(X)\to K$  is a function (not necesarelly definable) picking the centers of such cells. In addition, for each $\lambda=(J,a)\in \sigma_1(S_0)$, each $1\leq j\leq |J|$, each $s\in S_\lambda$ and $\delta_j\in J$ ($1\leq j\leq|J|$) the following holds \small
\[\exists y\in K\forall t\in K \left(C_{\delta_j}(t,y,\alpha_{\lambda 1j}(s),\alpha_{\lambda 2j}(s)) \rightarrow \left(f_s(t)=\alpha_{\lambda 3j}+a_j\frac{\ord(t-y)-k_{\delta_j}}{n_{\delta_j}}\right)\right).
\]
\normalsize
We stress again that we make no claims about the definability of the centers. 
\end{proof}




Since we work in a two-sorted structure, we will also encounter cells where the leading variable is a $\Gamma_K$-variable. For this case we have the following cell decomposition result:

\begin{prop}\label{prop:partialcd2}
%Let $(K, \Gamma_K, \Lm_2)$ be a relatively P-minimal structure.
%Let $S \subseteq K^m \times (\Gamma_K)^{m'}$, 
Let $X \subseteq S \times \Gamma_K$ a definable set, and $f: X \to \Gamma_K$ a definable function.% with elements $(s,\gamma)$, where $s \in S$ and $\gamma \in \Gamma_K$.  
There exists a finite decomposition of $X$ into $\Gamma_K$-cells $C$ of the form
\[C = \left\{(x,\gamma) \in D \times \Gamma_K \ \left| \ \begin{array}{l}\alpha(x)\ \square_1 \ \gamma \ \square_2 \  \beta(x)\quad \text{and}\quad \gamma \equiv n_0 \mod n \end{array} \right\}\right.,\]
where $D$ is an $\Lm_2$-definable subset of $S$, $\alpha, \beta: S \to \Gamma_K$ are definable functions, and $n_0,n \in \NN$. The partition can be taken such that on each part $C$, there exists a constant $a_{C} \in \ZZ$ and a definable function $\delta: D \to \Gamma_K$, such that for all $(x,\gamma) \in C$,
\[f(x,\gamma) = a_C \left(\frac{\gamma -n_0}{n}\right) + \delta(x).\]
 \end{prop}
\begin{proof}
The proof is similar to the proof of the previous theorem. By relative $P$-minimality, we know that for every $s \in \Pi_X(S)$, the fiber $X_s$ is a Presburger definable set. Cluckers \cite{clu-presb03} obtained a cell decomposition theorem for Presburger structures. Applying this to the sets $X_s$, yields that each
 $X_s$ can be partitoned into a finite union of cells of the form
\[C_s := \{ \gamma \in K \mid  \alpha_s\ \square_1 \ \gamma \ \square_2 \  \beta_s\quad \text{and}\quad \gamma \equiv n_0 \mod n_s\},\]
where $\alpha_s, \beta_s \in \Gamma_K$ and $n_s \in \NN$ are constants depending on $s$. 

Also note that for any $s \in \Pi_X(S)$, the graph of the function $f_s(\gamma):= f(s,\gamma)$ will be a presburger set, by the assumption of relative $P$-minimality. Indeed, $G_s:=\{(x,\gamma)\in K\times \Gamma_K: f_s(\ord(x))=\gamma)\}$ is $\Lm_{ring,2}$-definable by relative P-minimality. Therefore the set $G_s':=\{(x,y)\in K^2: (x,\ord(y))\in G_s\}$ is semi-algebraic. But 
\[Graph(f_s)=\{(\ord(x),\ord(y))\in \Gamma^2: (x,y)\in G_s'\},\]
which by Theorem \ref{thm:semialgpres} is Presburger definable. This means that each $f_s$ is a Presburger definable funtion, and hence  must be piecewise linear (with coefficients in $\QQ$). In particular, the above partition can be taken such that on each $C_s$, there exist constants $a_s \in \ZZ, \delta_s \in \gamma_K$ such that for all $\gamma \in C_s$, we have that
\[f_s(\gamma) = a_s \left(\frac{\gamma -n_0}{n_s}\right) + \delta_s.\]

Our claims follow then from a similar argument as the one presented in the proof of Theorem
\ref{thm:partialprep}. Since no centers are required in this case, the cell decomposition we obtain is even fully definable.

%We now claim that the number of cells needed to partition each set $X_s$ is bounded, i.e., 
% there exist $L,N \in \NN$, such that the following holds. Let \[P:= \{0, \ldots, N-1\} \times \{\emptyset, <\}^{2} \times \{1, \ldots, L\}\] be a set consisting of tuples $( \lambda, \Box, i)$.
% For every $s \in S$, one can find a subset $I$ of $P$, and corresponding tuples 
% $(d_j)_{j\in I}$, where
%  $ d_{j}:= (\alpha_{j}, \beta_{j}) \in \Gamma_K^{2}  ,$ 
% such that $X_s$ can be partitioned as $X_s = \bigcup_{j\in I} C_j(d_j)$. DO WE NEED TO EXPLAIN NOTATION $C_j$?? 
% 
% This can be shown using essentially the same compactness argument as on page \pageref{proof of claim}. Once this is established, one can proceed further in the same way as in the proof of Theorem \ref{thm:partialcd1}, with the obvious difference that no centers are required here.
% 
% {\color{blue}
% The function preparation can be achieved similarly as in the proof of Theorem \ref{thm:partialcd1}. In this case we start from the observation that whenever we fix an element $s$ of $S$, the graph of the function $f_s(\gamma):= f(s,\gamma)$ is a Presburger set, and hence $f_s$ must be Presburger definable, and hence be piecewise linear (with coefficients in $\QQ$). 
% }
\end{proof}
% !TEX root = nov.tex
\section{Integration}
In this section, $K$ denotes a $p$-adic field, so the value group $\Gamma_K$ will just be $\Z$. Two types of integrals will appear. When integrating over (subsets of) $\cO_K^m$, the Haar measure $\mu$ is used. When integrating over $\Z^n$, we use the counting measure.

We first recall the following Definition and Theorem from \cite{Clu-Gor-Hal-14}.
\begin{defn}Let $D \subseteq \ZZ^r$ be a Presburger set, $S$ an $\Lm_2$-definable set.
\item A function $f: D \to \AA_{q_K}$ is called Presburger constructible if it is contained in the $\AA_{q_K}$-algebra generated by Presburger functions $D\to \ZZ$, and functions $D\to \AA_{q_K}: x \mapsto q_K^{\beta(x)}$, where $\beta:D \to \ZZ$ is a Presburger function.
\item A function $f:X\subseteq S \times \ZZ^r$ is called Presburger constructible over $S$ if  the restrictions $f_s: X_s \to \AA_{q_K}$ are (uniformly) Presburger constructible for all $s \in S$. 
\item Given a function $h:D\to \AA_{q_K}$, we let $Z(h)$ denote the zero locus of $h$.
\end{defn}
\begin{def-theorem}[\cite{Clu-Gor-Hal-14}, Theorems 3.1.3 and 3.1.5]\label{thm:presburgerloci} Let $S$ be an $\Lm_2$-definable set, 
and $f:Y\subseteq S\times \ZZ^r\to \AA_{q_K}$ a function which is Presburger-constructible over $S$. Define the following sets:
\begin{align*}
\text{Iva}(f,S)&:=\{s \in S \mid f(s, \cdot) \text{ is identically zero on } Y_s\}\\
\text{Int}(f,S) &:=\{s \in S \mid f(s, \cdot) \text{ is measurable and integrable on } Y_s \}
\end{align*}
\item There exist $\Lm_2$-constructible functions $h_i: S \to \AA_{q_K}$, such that 
\[\text{Iva}(f,S) = Z(h_1), \text{\quad and \quad }
\text{Int}(f,S) = Z(h_2).\]
%\text{Bdd}(f,S) &= Z(h_2),\\
%.
%\end{align*}
\item There exists a function $g: Y \to \AA_{q_K}$, Presburger-constructible over $S$, such that $\Int(g,S) = S$ and $f(s,y) = g(s,y)$ whenever $s \in \text{Int}(f,S)$.
\end{def-theorem}
We will also need the following result
\begin{thm}\label{thm:presburger-int}
Let $S$ be an $\Lm_2$-definable set and $f: Y \subseteq S \times \ZZ^r \to \AA_{q_K}$ a function which is Presburger-constructible over $S$. If $\text{Int}(f,S) =S$, then there exists an $\Lm_2$-constructible function $g:S \to \AA_{q_K}$, such that for all $s \in S$,
\[ g(s) = \int_{Y_s} f(s,t)|dt|.\]  
\end{thm}
\begin{proof}
This is a reformulation of Theorem-Definition 4.5.1 in \cite{clu-loe-08}. The original theorem is for the case where $S$ is Presburger-definable, but the same proof applies in this context.
\end{proof}
We are now ready to state and prove our main theorem:
\begin{thm}
Let $(K, \Z, \Lm_2)$ be a relatively $P$-minimal structure, $S$ an $\Lm_2$-definable set and $f: X \subseteq S\times K^m \to \AA_{q_K}$ a constructible function.  There exists a constructible function $g: S \to \AA_{q_K}$, such that
\[g(s) = \int_{X_s} f(s,x)|dx|,\]
whenever $s \in \text{Int}(f,S)$.
\end{thm}
\begin{proof}
Since the result for general $m$ can be obtained by iteration, we may assume that $m =1$. A general constructible function has the form
\[f(s,x) = \sum_{i=1}^r a_i q_K^{f_{i0}(s,x)} \prod_{j=1}^{r'}f_{ij}(s,x),\]
where the $f_{ij}$ are definable functions $X \to \ZZ$, and $a_i \in \AA_{q_K}.$
Now put $\gamma = (\gamma_{ij})_{i,j}$ and consider the set
\[\text{GR}(f):= \{(s,\gamma, x)\in S\times \Gamma_K^{(r'+1)r}\times K \mid  \gamma_{ij} = f_{ij}(s,x)\},\]
which is a permutated version of the combined graphs of the functions generating $f$. We will partition this set in cells, using the cell decomposition results from the previous section. It is easy to see that an iteration of Theorem \ref{thm:partialprep} and Proposition \ref{prop:partialcd2} yields the following.

Put $R = r(r'+1)$. To ease the notation somewhat, we will sometimes renumber $\gamma =(\gamma_i)_{i=1\ldots R}$, where the correspondence is $\gamma_{ij}:= \gamma_{(i-1)(r'+1)+j+1}$.
Write $\langle \gamma\rangle_i := (\gamma_1, \ldots, \gamma_{i})$.
There exists a partition of $\text{GR}(f)$ into a finite union of cells $A$ of the following form. 
For each $A$, one has a definable set $\tilde{S}_A \subseteq S$, and for $1 \leqslant i \leqslant R+1$, a tuple of definable functions
\[\alpha_i: \tilde{S}_A \times \Gamma_K^{i-1} \to \Gamma_K: (s, \langle \gamma\rangle_{i-1}) \mapsto \alpha_i(s, \langle \gamma\rangle_{i-1}),\]
and similarly defined functions $\beta_i$. Put $\alpha := \alpha_{R+1}, \beta:= \beta_{R+1}$.  One also has a (not necessarily definable) function $c: \tilde{S}_A \times \Gamma_K^{R}\to K$.
For each $i =1, \ldots R$, let $\lambda_i =(\square_{i,1}, \square_{i,2},k,N) \in P_{\Gamma}$. We recall the notation 
\[
%B_{\lambda_i}(\alpha_i,\beta_i,\gamma_i):= \left[\begin{array}{l} \alpha_i(s,\langle \gamma\rangle_{i-1})\ \square_{i,1} \ \gamma_i \ \square_{i,2} \ \beta_i(s,\langle \gamma\rangle_{i-1}),  \ \ \ \wedge \\
% \gamma_i \equiv k_i\mod N \end{array}\right].
B_{\lambda_i}(\alpha_i,\beta_i,\gamma_i):=  [\alpha_i(s,\langle \gamma\rangle_{i-1})\ \square_{i,1} \ \gamma_i \ \square_{i,2} \ \beta_i(s,\langle \gamma\rangle_{i-1}) \ \wedge \
 \gamma_i \equiv k_i\mod N] .
\]
If we let $B$ be the set \[
B:= \left\{(s,\gamma) \in \tilde{S}_A \times \Gamma_K^R \ \left| \ \bigwedge_{i=1}^R B_{\lambda_i}(\alpha_i,\beta_i,\gamma_i)\right\}\right.,\]
then the partition of $\text{GR}(f)$ will consist of cells of the form
\[A:=\left\{(s, \gamma, x) \in B \times K \left| \begin{array}{l} 
\alpha(s,\gamma) \ \square_{1}\ \ord(x -c(s,\gamma)) \ \square_{2}\ \beta(s, \gamma),\\
\ord(x -c(s,\gamma)) \equiv k \mod N,\\ \ac_M(x -c(s,\gamma, )) = \xi
\end{array}
\right\}\right.,\]
for some $(\square, k, N, M, \xi) \in P_K$.

This partitioning can now be used to compute the given integral. Let $\mu$ denote the usual Haar measure (i.e., normalized such that $\mu(cO_K) =1$. ) We get the following:
\begin{eqnarray*}
 \int_{X_s} {f(s,x)}|dx|
 &=& \sum_{\delta \in \text{Im}(f_s)} \delta \cdot \mu\{x\in X_s \mid f_s(x) = \delta\}\\
 &=& \sum_\gamma  \left[\left(\sum_{i=1}^r a_i q_K^{\gamma_{i0}} \prod_{j=1}^{r'}\gamma_{ij}\right)\cdot \mu(\{x\in X_s \mid \bigwedge_{ij} f_{ij}(s,x) = \gamma_{ij}\}\right]\\ 
 &=& \sum_{\{A \mid s \in \tilde{S}_A\}} \left[ \sum_{\gamma \in B_s} \left(\sum_{i=1}^r a_i q_K^{\gamma_{i0}} \prod_{j=1}^{r'}\gamma_{ij}\right) \cdot \mu(A_{s,\gamma})\right].
\end{eqnarray*}
Let us now compute the measure of a fiber $A_{s,\gamma}$. %We will extend the notation $\langle \ldots\rangle_{i}$ introduced above, and write $\langle D\rangle_{i}$ for the projection of the elements of $D$ onto their first $i$ co\"ordinates.  For each $\langle x\rangle_{i-1} \in \langle A_{s,\gamma}\rangle_{i-1}$, write 
%\begin{align*}\Phi_{s,\gamma}(\langle x\rangle_{i-1})&:= \{ x_{i} \in K \mid x_i \text{ satisfies } \phi_{i1}(c_i) \wedge \phi_{i2}(c_i)\}\\
%\Phi^{0}_{s,\gamma}(\langle x\rangle_{i-1})&:= \{ x_{i} \in K \mid x_i \text{ satisfies } \phi_{i1}(0) \wedge \phi_{i2}(0)\}
%\end{align*}
Put $u =x - c(s, \gamma)$. Since the Haar measure is translation invariant, we have $|dx|=|du|$. Let $\hat{A}$ denote the set
\[\hat{A}:=\left\{(s, \gamma, u) \in B \times K \left| \begin{array}{l} 
\alpha(s,\gamma) \ \square_{1}\ \ord u \ \square_{2}\ \beta(s, \gamma),\\
\ord u \equiv k \mod N,\\ \ac_M(u) = \xi
\end{array}
\right\}\right.,\]
We get
\begin{eqnarray*}
\mu(A_{s,\gamma}) &=& \int_{A_{s,\gamma}}|dx|\\
&=& \int_{\hat{A}_{s,\gamma}}|du|
\\
&=& \sum_{\tau \in T}\mu(\xi \pi_K^{k + \tau N} (1 + \pi^{M}\cO_K)),\\
&=&|\xi|q_K^{-(k+M)}\sum_{\tau \in T} (q_K^{-N})^\tau,
\end{eqnarray*}
where $\pi_K$ is a uniformizing element of $K$ and $T$ is the set \[T:=\{\tau \in \Gamma_K \mid \alpha(s,\gamma ) \ \Box_{1}\ k + \tau N\ \square_{2}\ \beta(s,\gamma)\}.\]
Because of our assumptions, the set $A_{s,\gamma}$ must have finite measure, which is only possible if $\square_{1}$ denotes $<$.
(This is a property of the cell $A$, making it a definable condition)
 
We get the following results for this sum. 
%If possible, we will suppress the variables and just write $\alpha_m, \beta_m, \ldots$ to keep things readable. 
Put $\tilde{\alpha}:= \lfloor\frac{\alpha -k}{N}\rfloor+1$, and $\tilde{\beta}:= \lceil\frac{\beta -k}{N}\rceil-1$ (clearly these are still $\Lm_2$-definable functions). 

\begin{description}
\item[If $\square_{1}$ denotes \emph{no condition}]
\begin{equation}\label{eq:nocond}\sum_{\tau \in T} (q_K^{-N})^\tau = \frac{q_K^{-N\tilde{\alpha}}}{1-q_K^{-N}} \end{equation}
\item[If $\square_{1}$ denotes <]
\begin{equation}\label{eq:ineq}\sum_{\tau \in T} (q_K^{-N})^\tau = \frac{q_K^{-N\tilde{\alpha}}-q_K^{-N\tilde{\beta}}}{1-q_K^{-N}} \end{equation}
\end{description}
These are both constructible functions, so there exists a constructible function $h(s,\gamma)$ such that \[h(s,\gamma) = \mu(A_{s, \gamma}).\]
Hence, returning to the computation of $\int_{X_s} f(s,x)|dx|$, we need to compute a sum of the form
\[\sum_{\gamma \in B_s}\left[ \left(\sum_{i=1}^r a_i q_K^{\gamma_{i0}} \prod_{j=1}^{r'}\gamma_{ij}\right) \cdot h(s,\gamma)\right]\]
To compute this sum over gamma, we need to specify how $h(s,\gamma)$ depends on $\gamma$. 
By (an interation of) proposition \ref{prop:partialcd2}, we may assume that the variables $\gamma_i$ only occur in the functions $\alpha_i$, $\beta_i$ (in \eqref{eq:ineq} and \eqref{eq:nocond} ) in a {\color{red} linear} way (possibly after refining the cell decomposition). Hence, the problem is reduced to showing that a sum of the form 
\[\sum_{\gamma \in B_s} \sum_{i=1}^r \left(a_i q_K^{\nu_i(s)}q_K^{b_i(\gamma)} \prod_{j=1}^{R}\gamma_{j}^{a_{ij}}\right),\]
%$\sum_{\gamma \in \Pi_{\gamma}(A_s)} q_K^{a\gamma + \delta(s)} $ 
where the $b_i$ are Presburger definable linear functions, %(i.e., of the form $\sum_{j=1}^R{\mu_{ij}\left(\frac{\gamma_{i}-k_i}{N}\right)}$)
$a_{ij} \in \mathbb{N}$ and $a_i \in \AA_{q_K}$, can be computed uniformly.
Let $F: B\to \AA_{q_K}$ be the function defined by
\[F(s,\gamma) = \left\{\begin{array}{ll}{\sum_{i=1}^r \left(a_i q_K^{\nu_i(s)}q_K^{b_i(\gamma)} \prod_{j=1}^{R}\gamma_{j}^{a_{ij}}\right)} & \text{if } (s,\gamma)\in B\\ 0 & \text{otherwise}.
\end{array}.\right.\]
This function is Presburger constructible over $S$, so we can apply
Theorem-Definition \ref{thm:presburgerloci} to conclude that its locus of integrability $\text{Int}(F,S)$ can be defined uniformly as the zero locus of an $\Lm$-constructible function $h: S\to \AA_{q_K}$. 
This lemma also implies that one can find a function $G(s,\gamma)$ with $\text{Int}(G,S)= S$ and $G(s,\gamma) = F(s,\gamma)$ whenever $s \in \text{Int}(F,S)$. Because of this, we may as well assume that $\text{Int}(F,S)=S$. Applying Theorem \ref{thm:presburger-int} to $G(s,\gamma)$ concludes the proof.
\end{proof}
%\begin{defn}

%\end{defn}

% !TEX root = nov.tex
\section{Rationality}
In this section we will give a proof of Theorem \ref{thm:rationality}, which we restate here in a version that gives more details about the exact form of the rational function. 
\begin{thm}
Suppose that $(K,\cL)$ is relatively P-minimal. Let $X$ be an $\cL_2$-definable subset of $D\times \NN$, where $D$ is a compact subset of $K^m$.  Write $a_n$ for the Haar measure of $X_n:=\{x\in K^m\mid (x,n)\in X\}$, for each $n\geq 0$. Then the series $\sum_{i\geq 0} a_i T^i$ is a rational function of the form
\[\sum_{i\geq 0} a_i T^i = \frac{Q(T) }{(1-T^N)^k},\]
where $k, N \in \N$ and  $Q(T) \in \AA_{q_K}[T]$. 
\end{thm}
\begin{proof}


Applying Theorem \ref{thm1} to the set $X \subset \cO_K^n\times \NN$, one can find a constructible function $g: \NN \to \AA_{q_K}$, such that
\[g(n) = \int_{X_n}|dx|.\]
This function must have the form 
\[g(n) = \sum_{i=1}^r a_iq_K^{\alpha_i(n)}\prod_j\beta_{ij}(n),\]
where $a_i \in \AA_{q_K}$, and the functions $\alpha_i$ and $\beta_{ij}$ are Presburger-definable functions $\NN \to \ZZ$. By the Preparation theorem for Presburger functions, we can partition $\N$ in $\Gamma$-cells $C_1, \ldots C_r \subseteq \N$ of the form
\[C_i = \{ n \in \N \mid A_i \ \square_1 \ n \ \square_2\ B_i \ \wedge\ n \equiv k_i \mod N \}\]
 such that on each part $g_i:= g_{|C_i}$ has the form 
\[g_i(n) = \sum_{j=1}^r a_{ij}q_K^{m_{ij}(\frac{n-k_i}{N})}P_i\left(\frac{n-k_i}{N}\right),\]
where $P_i(t)$ is in $\Z[t]$, $m_{ij} \in \N$ and $a_{ij} \in \AA_{q_K}$. Moreover, there exists $M \in \N$, such that $g_i(n) <M$ for all $i,n$, since each $X_n$ is contained in the compact set $D$.
 
Now put $n_i' := \frac{n-k_i}{N}$, and write $C_i'$ for the set
\[C_i'=\{n_i'\in \N \mid A_i'\ \square_1 \ n' \ \square_2 \ B_i'\},\]
where $A_i' =\lfloor \frac{A_i-k_i}{N}\rfloor$, and $B_i' =\lceil \frac{B_i-k_i}{N}\rceil$. Write $I = \{1, \ldots, r\}$, and let $I_1$ denote the set of all $C_i'$ be a set on which $\square_2$ denotes \emph{no condition}. It is easy to check that the condition $g_i'(n_i) <M$ must imply that $m_{ij}<0$ whenever $i \in I_1$.  

We will now consider the integral
\[I(\sigma):=\sum_{n\in\NN}(q_K^{-\sigma})^{n}\cdot \int_{X_n}|dx|,\] 
and show that $I(\sigma)$ is rational in $q_K^{-\sigma}$.

Using the observations above, we get that
\begin{eqnarray*}
I(\sigma) %&=& \sum_{n\in\NN}(q_K^{-\sigma})^{n}\cdot \int_{X_n}|dx|\\
&=& \sum_{i\in I}\left[ \sum_{n \in C_i} (q_K^{-\sigma})^{n}g_i(n)\right]\\
&=&\sum_{C_i'}(q_K^{-\sigma})^{k_i} \sum_{n_i' \in C_i'}\sum_{j=1}^r a_{ij}'q_K^{(-N\sigma +m_{ij})n_i'}P(n_i').
\end{eqnarray*}
%where $C_i'$ is the set $\{n'\in \N \mid A_i'\ \square_1 \ n' \ \square_2 \ B_i'\}$. 
This sum can be computed, using the fact that for all $a \in \N$ and $P(t) \in \Z[t]$ of degree $d$, the following holds in $\Z[[t]]$:
\[\sum_{n \geqslant a} P(n)t^n =\sum_{i=0}^d	\frac{[\Delta^iP(a)]t^{a+i}}{(1 - t )^{i+1}}.\]
Here $\Delta^i$ is the $i$-th iterate of the difference operator $\Delta:P \mapsto P(X + 1) - P(X)$ with the convention $\Delta^0P = P$.

{\color{red} This is an equality for formal power series. But the computation includes sums $\sum (q_K^{m_{ij}})^{n'}P(n')$, and these need to be convergent, so we need that $m_ij <0$. Can this be deduced from the fact that $g_i(n)<1$ (since it represents the measure of a subset of $\mathcal{O}_K^n$)??? }
 
In particular, it is clear that $I(\sigma)$ is rational in $T:=q_K^{-\sigma}$. The specific form of the fraction follows easily from the calculation.
\\\\\\\\\\
%Let $\alpha$ denote the projection map $\alpha:X\to \NN: (x,n) \mapsto n$, and consider the integral
%\[I(\sigma):= \int_X (q_K^\alpha)^{-\sigma}|dn||dx|,\]
%where $\sigma \in \RR$, with $\sigma>0$. Note that  $I(\sigma) =\sum_{n\in\NN}(q_K^{-\sigma})^{n}\cdot \int_{X_n}|dx|$, and hence we need \end{proof}
\appendix
% !TEX root = nov.tex
\section{Definably well-ordered structures}\label{appendix}

\begin{defn}
An $L$-structure $M$ is said to be\emph{definably well-ordered} if there is an definable linear order $\lhd$ on $M$ such that any definable subset of $M$ has a $\lhd$-minimal element. 
\end{defn}

\begin{lem}\label{lem.elem} Suppose that $M$ is a definably well-ordered $L$-structure and let $\lhd$ be defined by an $L(a)$-formula. Then for every $L(a)$-structure $N$, if $N\equiv_a M$ then $N$ is definably well-orderable. 
\end{lem}

\begin{proof}
Let $\phi(x,y)$ be an $L(a)$-formula with $length(x)=1$. By definition 
\[M\models \forall y (\exists x (\phi(x,y))\to \exists x (\phi(x,y)\wedge \forall z (\phi(z,y)\to x\unlhd z))).
\]
Since $N\equiv_a M$, this implies that every definable subset of $N$ has a $\lhd$-minimal element.
\end{proof}

The previous lemma shows that being a definably well-ordered structure is a property of $Th(M,a)$, where $a$ is a tuple of parameters used in a formula defining a linear order such that any definable subset of $M$ has a minimal element. We say that a theory $T$ is definably well-orderable if it has some definably well-ordered model where the linear order is 0-definable. The following lemma shows the relation with cartesian powers:  

\begin{lem}\label{lem.cartesian} The following are equivalent:
\begin{enumerate}
	\item $M$ is definably well-ordered; 
	\item there is an $L(M)$-definable linear order $\lhd$ on $M^n$ such that any definable subset of $M^n$ has a $\lhd$-minimal element.
\end{enumerate}
\end{lem}

\begin{proof}
That (1) implies (2) follows by putting the lexicographic on $M^n$ induced by the definable linear order on $M$. For the converse, suppose that $n>1$ and pick any $a\in M^{n-1}$. Let $\phi(x,y)$ be a formula defining $x\lhd y$. Then $\phi(x_1,a;y_1,a)$ defines a well-order on $M$.  
\end{proof}

In light of \ref{lem.cartesian}, given a definably well-ordered structure $M$ we denote by $\lhd$ a fixed definable linear-order on $M$ such that definable sets in any cartesian power of $M$ have $\lhd$-minimal elements. For a theory to have models which are definably well ordered is a very strong property. As an example we show that such theories have definable choice and thus, they eliminate imaginaries. 

\begin{prop}
An definably well-ordered structure $M$ has definable choice. 
\end{prop}

\begin{proof}
Let $X\subseteq M^{m+n}$ be a definable set. Define $f:\Pi_m(X)\rightarrow M^n$ to be the function sending $x$ to the $\lhd$-least element in $X_x$. Clearly, if $X_x=X_y$ then $f(x)=f(y)$. 
\end{proof}

\begin{cor}
A definably well-ordered structure $M$ has definable Skolem functions. 
\end{cor}

\begin{cor}
A definably well-ordered structure $M$ has uniform elimination of imaginaries. 
\end{cor}

Notice that being a definably well-ordered structure is stronger than to have definable choice. For instance the real field has definable choice but by a result of Ramakrishnan in \cite{ramakrishnan-12} every definable order embeds in $(\RR^n,<_{\text{lex}})$. Therefore, no definable linear order has minimal elements for all definable subsets of the real line. Using this one can show that no reduct the real field is definably well orderable. 

\

Though definably well-ordered structures have strong model-theoretic properties, they are not always model-theoretically tame. For instance, the theory of arithmetic is definably well-orderable and yet model-theoretically wild. The main example of a tame well-orderable theory is Presburger Arithmetic. This is a consecuence of the following proposition: 

\begin{prop}\label{preswellorder}
Let $\Lm$ be a language containing $\{\leq,-\}$ (as $\Lm_{Pres}$). Then $Th(\ZZ,\Lm)$ is definably well-orderable.
\end{prop}

\begin{proof}
Consider the following definable order

\[x\lhd y \Leftrightarrow 
\begin{cases}
0\leq x<y \\
0\leq x<-y \\
0\leq -x<y\\
0\leq -x<-y \\
\end{cases}
\]

On $\ZZ$ this actually defines the following well-order: 
$$0 \lhd 1 \lhd -1 \lhd 2 \lhd -2 \lhd \cdots  .$$

In light of lemma \ref{lem.elem}, this completes the proof. 
\end{proof}

Notice that the linear ordering defined in the previous proposition does not define in general a well-order on a $\ZZ$-group $G$, but it does define a linear order such that for any definable subset $A\subseteq G$, $A$ has a $\lhd$-minimal element. As a corollary we get a result from \cite{clu-presb03} 

\begin{cor}
Presburger arithmetic has elimination of imaginaries. 
\end{cor}

\bibliographystyle{amsplain}
\bibliography{../Bibliografie}

\end{document}


