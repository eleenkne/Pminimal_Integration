% !TEX root = nov.tex
\section{Introduction}

Let $K$ be a $p$-adic field and $\Gamma_K$ be its value group. Let $\cL_{\rm ring}$ be the ring language and let $\cL$ be a language containing $\cL_{\rm ring}$. Associated to $\cL$ is the two-sorted language $\cL_2$ which consists of a sort for $K$ in $\cL$, a sort for $\Gamma_K\cup\{+\infty\}$ in the language of Presburger arithmetic $\cL_{Pres}$ and the valuation map $\ord: K\to \Gamma_K\cup\{+\infty\}$. In order to prove the rationality of different functions, Denef introduced the following class of functions: 

\begin{defn}
Let $X$ be an $\cL_2$-definable set. Write $\AA_{q_K}$ for the ring \[\AA_{q_K}:=\ZZ\left[q_K, q_k^{-1}, \left(\frac1{1-q_k^{-i}}\right)_{i\in \NN, i>0}\right].\]
Call a function $f:X\to \QQ$ $\cL$-constructible (or $\cL_2$-constructible) if it is contained in the $\AA_{q_K}$-algebra generated by functions of the forms
\begin{enumerate}
\item $\alpha:X\to \ZZ$
\item $X\to \ZZ:x\mapsto q_K^{\beta(x)}$,
\end{enumerate}
where $\alpha$ and $\beta$ are $\cL_2$-definable and $\ZZ$-valued.  
\end{defn}

When $\cL$ is $\cL_{\rm ring}$, the subanalytic language $\Lm_{an}$ on $K$ (see \ref{eq:subanalytic}) or some intermediary languages as in \cite{CLip}, the class of $\Lm$-constructible functions are known to be stable under integration (see \cite{denef-2000}, \cite{Clu-2003}, and \cite{Clu-Gor-Hal-14} for the most convenient dealing with integrability conditions). The analogous class of functions, build up from definable functions in a general $P$-minimal structure is not known to have this stability property. In this article we suggest a notion of \emph{relative $P$-minimality}, for which we show stability under integration. We prove a preparation theorem based on cell-decomposition which is the main ingredient in the proof for the semi-algebraic and subanalytic cases. Such preparation theorem is not known to be true in the $P$-minimal context. Mourgues \cite{mou-09} proved a somewhat weaker form of cell decomposition under the assumption of Skolem functions on $(K,\cL)$, but this form of cell decomposition seems not enough to show stability under integration. The idea of relative $P$-minimality is to bring into account the value group in the definition of minimality as follows:

\begin{defn}
Call the $\Lm_2$-structure $(K,\Gamma_K)$ \emph{relatively $P$-minimal} (relative to the sort $\Gamma$), if, for any elementary equivalent $\Lm_2$-structure $(K',\Gamma_{K'})$, any $\Lm_2$-definable subset $X\subseteq K'\times \Gamma_{K'}$ is already $\cL_{\rm ring, 2}$-definable.
\end{defn}
Here, $\Gamma_{K'}$ stands for the value group of $K'$. When $\cL$ is $\cL_{\rm ring}$ or $\Lm_{an}$, the $\Lm_2$-structure $(K,\Gamma_K)$ is relative $P$-minimal (as we will show in Lemma \ref{lem:suban}). The preparation theorem that we show is the following:  

\begin{thm}[Preparation theorem for relative $P$-minimality]\label{thm:partialprep} 
Let $(K,\Gamma_K)$ be relatively $P$-minimal $\Lm_2$-structure. Let $S\subseteq K^l\times\Gamma^l$ and $X \subseteq S \times K$ be $\Lm_2$-definable sets, and $f:X\rightarrow \Gamma_k$ be an $\Lm_2$-definable function. There exists a finite decomposition of $X$ into cells $C$ of the form  
\[C = \left\{(s,t) \in D\times K \ \left| \ \begin{array}{l} \alpha(s) \square_1 \ \ord(t-c(s)) \ \square_2 \ \beta(s),\\ \ord(t-c(s)) \equiv k\mod n,\\ \ac_{m}(t-c(s)) = \xi \end{array} \right\}\right..\]
On each such cell $C$, there is a $\Lm_2$-definable function $\gamma:D\to\Gamma_K$ and an integer $a\in \ZZ$ such that for all $(s,t) \in C$, 
\[f(s,t) = \gamma(s) + a\frac{\ord(t-c(s))-k}{n}.\]
We stress that the functions $c: S\to K$ are NOT necessarily $\Lm_2$-definable.
\end{thm}

As an application we are able to show (for a more precise statement see \ref{thm:integration}): 

\begin{thm*}
Let $(K, \Z)$ be a relatively $P$-minimal structure, $S$ an $\Lm_2$-definable set and $f: X \subseteq S\times K^m \to \AA_{q_K}$ a constructible function.  There exists a constructible function $g: S \to \AA_{q_K}$, such that
\[g(s) = \int_{X_s} f(s,x)|dx|,\]
whenever $f(s, \cdot)$ is measurable and integrable on the fiber $Y_s$.
\end{thm*}

We also extend the rationality results so far known only for the semi-algebraic \cite{denef-84} and subanalytic setting \cite{denef-vdd-88} (and thus also for any sublanguage), obtaining the follwing:

\begin{thm}\label{thm:rationality}
Suppose that $(K,\cL)$ is relatively $P$-minimal. Let $X$ be a $\cL_2$-definable subset of $\cO_K^n\times \NN$, and let $a_n$ be the Haar measure of $X_n:=\{x\in \cO_K^n\mid (x,n)\in X\}$ for each $n\geq 0$. Then the series $\sum_{i\geq 0} a_i T^i$ is rational. 
\end{thm}
Here, we normalize the Haar measure on $K^n$ so that $\cO_K^n$ has measure $1$. For a more general version see Theorem \ref{thm:rationalitygeneral}.  

\

The article is presented as follows. Section 2 is devoted to prove the preparation theorem for relative $P$-minimal structures. Sections 3 and 4 correspond respectively to the integration and rationality results. Finally, in section 5 we discuss generalities about the notion of relative $P$-minimality and its relation with classical $P$-minimality. Our arguments will use a result on definable sets of Presburger arithmetic that we include an appendix about ordered structures. 

\subsection{Preliminaries}

Let $(K,\Gamma_K)$ be a $\cL_2$-structure. By a definable set we mean an $\Lm_2$-definable set allowing parameters. For notational purposes we will fix a definable set $S \subseteq K^{m_0} \times \Gamma_K^{m_0}$ which we call a \emph{parameter set}. For a definable set $X\subseteq S\times K$ and $s\in S$, $X_s:=\{x\in K: (s,x)\in X\}$ denotes the fiber over $s$. Analogously, for a definable function $f:X\rightarrow \Gamma_K$, $f_s$ denotes the function $f(s,\cdot):X_s\rightarrow \Gamma_K$. Given two sets $A$ and $B$, we denote by $\Pi_{A}:A\times B\to A$ for the projection onto $A$ and by $\Pi_B:A\times B\to B$ the projection onto $B$. For a positive integer $n\geq 1$, $A^{\leq n}$ denotes $\bigcup_{i=1}^n A^i$. We start defining cells in this context: 

\begin{defn}[Cells]\label{def:cell} Let $(K,\Gamma_K)$ be a $\cL_2$-structure.  
\begin{itemize}[leftmargin=*]
\item A subset $C\subseteq S\times K$ is a $K$-cell if it is of the form 
\[C = \left\{(s,t) \in D\times K \ \left| \ \begin{array}{l} \alpha(s) \square_1 \ \ord(t-c(s)) \ \square_2 \ \beta(s),\\ \ord(t-c(s)) \equiv k\mod n,\\ \ac_{m}(t-c(s)) = \xi \end{array} \right\}\right.,\]
where $D$ is a definable subset of $S$, $c$ is a function $c:D\to K$, $\alpha, \beta$ are $\Lm_2$-definable functions $D\to\Gamma_k$, $k, n, m \in \NN$, $\xi \in \ac_{m}(K)$ and the squares $\square_i$ may denote $<$ or \emph{no condition}. If the function $c$ is definable, we say that the cell has \emph{definable centers}.  
\item A subset $B\subseteq S\times \Gamma_K$ is a $\Gamma$-cell if it is of the form
\[B= \left\{(s,\gamma)\in D\times \Gamma_K \left|\begin{array}{l} \alpha(s) \square_1 \ \gamma \ \square_2 \ \beta(s), \\
\gamma \equiv k\mod n \end{array}\right\}\right.,\]
where $D$ is an definable subset of $S$, $\alpha, \beta$ are definable functions $D\to\Gamma_k$, $k, n\in \NN$ and again the squares $\square_i$ may denote $<$ or \emph{no condition}.  
\end{itemize}
\end{defn}

We call \emph{cell} a subset which is either a $K$-cell or a $\Gamma$-cell. Given cells $C$ and $B$ as in the previous definition we call the tuples $(\square,k,n,m,\xi)$ and $(\square,k,n)$ \emph{the type} of $C$ and $B$ respectively. We denote by $P_K$ (resp. $P_\Gamma$) the set of all possible types of $K$-cells (resp. $\Gamma$-cells), that is, the set of tuples $(\square,k,n,m,\xi)$  (resp. $(\square,k,n)$) where $\square = (\square_1, \square_2) \in \{\emptyset, <\}^{2}$, $(k,n,m)$ its a triple of positive integers such that $k<n$, and $\xi\in \ac_{m}(K)$. For $\delta=(\square,k,n,m,\xi)\in P_K$, $\lambda=(\square,k,n)\in P_{\Gamma}$, $\alpha,\beta,\gamma$ variables of type $\Gamma$ and $x,y$ variables of type $K$, we define the formulas
\begin{equation}\label{defKcells}
C_\delta(x,y,\alpha,\beta):= \left(\begin{array}{l} \alpha \square_1 \ \ord(x-y) \ \square_2 \ \beta \ \wedge\\ \ord(x-y) \equiv k\mod n \ \wedge\\ \ac_{m}(x-y) = \xi \end{array} \right),
\end{equation}
\begin{equation}\label{defGammacells}
B_\lambda(\alpha,\beta,\gamma):= \left(\begin{array}{l} \alpha \square_1 \ \gamma \ \square_2 \ \beta \ \wedge \\
\gamma \equiv k\mod n \end{array}\right).
\end{equation}

Notice that for a definable subset $C\subseteq S\times K$, we can now express the fact that $C$ is a cell with definable centers claiming the existence of definable functions $\alpha,\beta:\Pi_S(C)\to\Gamma_K$, $c:\Pi_S(C)\to K$ and of a tuple $\delta\in P_K$ such that 
\[C = \left\{(s,t) \in \Pi_S(C)\times K \ \left| \ (K,\Gamma_K)\models C_\delta(t,c(s),\alpha(s),\beta(s))\right\}\right..
\]
Analogously, for a definable subset $B\subseteq S\times \Gamma_K$, $B$ is a cell if there are definable functions $\alpha,\beta:\Pi_S(C)\to\Gamma_K$ and an tuple $\lambda\in P_\Gamma$ such that  
\[B = \left\{(s,\gamma) \in \Pi_S(B) \times \Gamma \ \left| \ (K,\Gamma_K)\models B_\lambda(\alpha(s),\beta(s),\gamma)\right\}\right..
\]



