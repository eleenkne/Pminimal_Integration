% !TEX root = nov.tex
\section{Introduction}


Let $K$ be a $p$-adic field, $\Gamma_K$ be its value group, $\cO_K$ its valuation ring, $\cM_K$ its maximal ideal and $q_K$ the number of elements of its residue field $k_K$. Write $\pi$ for a uniformizer of $K$. Let $\cL_{\rm ring}$ be the ring language and let $\cL$ be a language containing $\cL_{\rm ring}$ such that $(K,\cL)$ is P-minimal. Associated to $\cL$ is the two-sorted language $\cL_2$ which consists of a sort for $K$ in $\cL$, a sort for a sort for $\ZZ\cup\{+\infty\}$ in the language of Presburger arithmetic $\cL_{Pres}$ and the valuation map $\ord: K\to \ZZ\cup\{+\infty\}$. Denef introduced a certain class of functions, which has been shown to be stable under integration in the semi-algebraic and in the subanalytic case, see \cite{denef-2000}, \cite{Clu-2003}, and \cite{Clu-Gor-Hal-14} for the most convenient dealing with integrability conditions. The analogous class of functions, build up from our general P-minimal structure is not known to have this stability property. Here, we suggest a notion of relative P-minimality, for which we show stability under integration.

\begin{defn}

Call the structure $(K,\cL)$ relatively P-minimal (relative to the sort $\ZZ$), if, for any $(K',\cL)$ which is elementary equivalent to $(K,\cL)$, any $\cL_2(K')$-definable subset $X$ of $K'\times \Gamma_{K'}^\ell$ is already $\cL_{\rm ring, 2}(K)$-definable.

\end{defn}
Here, $\Gamma_{K'}$ stands for the value group of $K'$.

\begin{remark - ques}\label{rq1} Maybe the above definition with $\ell=1$ instead of general $\ell$ is enough, or maybe even equivalent to the version for general $\ell$. 
\end{remark - ques}


\begin{remark}\label{r1} Maybe it is convenient to also require definable skolem functions at some point, but one of the goals is to avoid such condition. 
\end{remark}


\begin{defn}
Let $X$ be an $\cL_2$-definable set. Write $\AA_{q_K}$ for the ring \[\AA_{q_K}:=\ZZ\left[q_K, q_k^{-1}, \left(\frac1{1-q_k^{-i}}\right)_{i\in \NN, i>0}\right].\]
Call a function $f:X\to \QQ$ $\cL$-constructible (or $\cL_2$-constructible) if it is contained in the $\AA_{q_K}$-algebra generated by functions of the forms
\begin{enumerate}
\item $\alpha:X\to \ZZ$
\item $X\to \ZZ:x\mapsto q_K^{\beta(x)}$,
\end{enumerate}
where $\alpha$ and $\beta$ are $\cL_2$-definable and $\ZZ$-valued.  
\end{defn}


When $\cL$ is $\cL_{\rm ring}$ or the subanalytic language on $K$ (or some intermediary languages as in \cite{CLip}), the classes of constructible functions are known to be stable under integration. This is proved using a form of preparation theorem for constructible functions, based on cell decomposition. In P-minimal structures, there are several difficulties related to cell decomposition. Mourgues \cite{mou-09} proved a somewhat weaker form of cell decomposition under the assumption of having definable Skolem functions on $(K,\cL)$, but this form of cell decomposition seems not enough to show stability under integration.

Here we investigate the following.

\begin{thm}\label{thm1}
Suppose that $(K,\cL)$ is relatively P-minimal. Then the class of $\cL$-constructible functions is stable under integration, in the sense of Section 3 of \cite{Clu-Gor-Hal-14}.
\end{thm}

Also rationality results are so far only known for the semi-algebraic setting \cite{denef-84}, the subanalytic setting \cite{denef-vdd-88} (and thus also for any sublanguage).

As a side result we also obtain the following.

\begin{thm}\label{thm:rationality}
Suppose that $(K,\cL)$ is relatively P-minimal. Let $X$ be a $\cL_2$-definable subset of $\cO_K^n\times \NN$, and let $a_n$ be the Haar measure of $X_n:=\{x\in \cO_K^n\mid (x,n)\in X\}$ for each $n\geq 0$. Then the series $\sum_{i\geq 0} a_i T^i$ is rational. 
\end{thm}
Here, we normalize the Haar measure on $K^n$ so that $\cO_K^n$ has measure $1$. 

\

Let $(K,\Gamma_K)$ be a $\cL_2$-structure. For notational purposes we will fix a definable set $S \subseteq K^{m_0} \times \Gamma_K^{m_0}$ which we call a \emph{parameter set}. For a definable set $X\subseteq S\times K$ and $s\in S$, $X_s:=\{x\in K: (s,x)\in X\}$ denotes the fiber over $s$. Analogously, for a definable function $f:X\rightarrow \Gamma_K$, $f_s$ denotes the function $f(s,\cdot):X_s\rightarrow \Gamma_K$. Given two sets $A$ and $B$, we denote by $\Pi_{A}:A\times B\to A$ for the projection onto $A$ and by $\Pi_B:A\times B\to B$ the projection onto $B$. For a positive integer $n\geq 1$, $A^{\leq n}$ denotes $\bigcup_{i=1}^n A^i$. We start defining cells in this context: 


\begin{defn}[Cells] Let $(K,\Gamma_K)$ be a $\cL_2$-structure.  
\begin{enumerate}
\item A subset $C\subseteq S\times K$ is a $K$-cell if it is of the form 
\[C = \left\{(s,t) \in D\times K \ \left| \ \begin{array}{l} \alpha(s) \square_1 \ \ord(t-c(s)) \ \square_2 \ \beta(s),\\ \ord(t-c(s)) \equiv k\mod n,\\ \ac_{m}(t-c(s)) = \xi \end{array} \right\}\right.,\]
where $D$ is a $\cL_2$-definable subset of $S$, $c$ is a function $c:D\to K$, $\alpha, \beta$ are $\Lm_2$-definable functions $D\to\Gamma_k$, $k, n, m \in \NN$, $\xi \in \ac_{m}(K)$ and the squares $\square_i$ may denote $<$ or \emph{no condition}. If the function $c$ is $\cL_2$-definable, we say that the cell has \emph{definable centers}.  
\item A subset $B\subseteq S\times \Gamma_K$ is a $\Gamma$-cell if it is of the form
\[B= \left\{(s,\gamma)\in D\times \Gamma_K \left|\begin{array}{l} \alpha(s) \square_1 \ \gamma \ \square_2 \ \beta(s), \\
\gamma \equiv k\mod n \end{array}\right\}\right.,\]
where $D$ is an $\cL_2$-definable subset of $S$, $\alpha, \beta$ are $\Lm_2$-definable functions $D\to\Gamma_k$, $k, n\in \NN$ and again the squares $\square_i$ may denote $<$ or \emph{no condition}.  
 langauge. 
\end{enumerate}
\end{defn}

We call \emph{cell} a subset which is either a $K$-cell or a $\Gamma$-cell. Using this terminology, a semi-algebraic cell is a cell with respect to the language $\cL_{ring,2}$ and a sub-analytic cell is a cell with respect to $\cL_{an,2}$, where $\cL_{an}$ is the sub-analytic language on $K$, i.e., Macintyre's language enriched with the field inverse $^{-1}$ on $K$ extended by $0^{-1}=0$ and, for each convergent power series $f: \cO^n
\to K$, a function symbol for the
restricted analytic function
\[x \mapsto 
\begin{cases}
f(x) &\text{ if }x \in \cO^n\\
0 &\text{ otherwise }.
\end{cases}\]
We will fix some notation and terminology regarding cells that will be used through the rest of the article. Let $\square = (\square_1, \square_2) \in \{\emptyset, <\}^{2}$ (we treat $\emptyset$ as ``no-condition'') $(k,n,m)$ be a triple of positive integers such that $k<n$ and $\xi\in \ac_{m}(K)$. Let $P_K$ denote the set of all tuples $(\square,k,n,m,\xi)$ and $P_\Gamma$ be the set of all tuples $(\square,k,n)$. For $\delta=(\square,k,n,m,\xi)\in P_K$, $\lambda=(\square,k,n)\in P_{\Gamma}$, $\alpha,\beta,\gamma$ variables of type $\Gamma$ and $x,y$ variables of type $K$, we define the formulas
\iffalse
\begin{align}\label{defcells}
\begin{split}
C_{1,\delta}(x,y,\alpha,\beta)&:= (\alpha\square_{1}\ \ord(x -y)\\square_{2}\ \beta)\\
C_{2,\delta}(x,y)&:= ({\ord(x-y)\equiv k \ \text{mod } n})\\
C_{3,\delta}(x,y)&:= (\ac_{m}(x-y)=\xi)\\
C_\delta(x,y,\alpha,\beta)&:=C_{1,\delta}(x,y,\alpha,\beta)\wedge C_{2,\delta}(x,y)\wedge C_{3,\delta}(x,y)\\
\\
B_{1,\lambda}(\alpha,\beta,\gamma)&:= (\alpha\square_{1} \gamma \ \square_{2}\ \beta)\\
B_{2,\lambda}(\gamma)&:= ({\gamma\equiv k \ \text{mod } n})\\
B_{\lambda}(\alpha,\beta,\gamma)&:=B_{1,\lambda}(\alpha,\beta,\gamma)\wedge B_{\lambda}(\gamma).
\end{split}
\end{align}
\fi

\begin{equation}\label{defKcells}
C_\delta(x,y,\alpha,\beta) = \left(\begin{array}{l} \alpha \square_1 \ \ord(x-y) \ \square_2 \ \beta \ \wedge\\ \ord(x-y) \equiv k\mod n \ \wedge\\ \ac_{m}(x-y) = \xi \end{array} \right),
\end{equation}
\begin{equation}\label{defGammacells}
B_\lambda(\alpha,\beta,\gamma):= \left(\begin{array}{l} \alpha \square_1 \ \gamma \ \square_2 \ \beta \ \wedge \\
\gamma \equiv k\mod n \end{array}\right).
\end{equation}

Notice that for a definable subset $C\subseteq S\times K$, we can now express the fact that $C$ is a cell with definable centers claiming the existence of definable functions $\alpha,\beta:\Pi_S(C)\to\Gamma_K$, $c:\Pi_S(C)\to K$ and of a tuple $\delta\in P_K$ such that 
\[C = \left\{(s,t) \in \Pi_S(C)\times K \ \left| \ (K,\Gamma_K)\models C_\delta(t,c(s),\alpha(s),\beta(s))\right\}\right..
\]
Analogously, for a definable subset $B\subseteq S\times \Gamma_K$, $B$ is a cell if there are definable functions $\alpha,\beta:\Pi_S(C)\to\Gamma_K$ and an tuple $\lambda\in P_\Gamma$ such that  
\[B = \left\{(s,\gamma) \in \Pi_S(B) \times \Gamma \ \left| \ (K,\Gamma_K)\models B_\lambda(\alpha(s),\beta(s),\gamma)\right\}\right..
\]

\subsection{$P$-minimality versus relative $P$-minimality}
How does our notion of relative $P$-minimality relate to the classical notion, as introduced by Haskell and Macpherson \cite{has-mac-97}?

When working in a $p$-adic setting, it is quite natural to distinguish the value group as a separate sort. Cluckers \cite{clu-presb03} made the following observations in this context:
\begin{thm}\label{thm:semialgpres}
Let $(K, \Lm)$ be a $P$-minimal field with $P$-minimal theory. 
\item For any $\Lm$-definable set $X \subseteq (K^{\times})^m$, the set
\[\ord(X):=\{(\ord x_1, \ldots, \ord x_m) \mid (x_1, \ldots, x_m) \in X\}\]
is $\Lm_{\text{Pres}}$-definable.
\item Let $S \subseteq \Gamma_K^m$ be a Presburger-definable set. Then the set
\[\{\ord^{-1}(S):= \{ (x_1, \ldots, x_m) \in X \mid \ord x \in S\}\]
is $\Lring$-definable.
\end{thm} 
Hence, the Presburger language on the value group sort is induced by $\Lring$ on the field sort, and the assumption of $P$-minimality ensures that this is sufficient even for bigger $\Lm$. Hence, it is quite natural to also impose minimality conditions on the value group sort. For example, the structure of subanalytic sets will satisfy this new minimality notion:

\begin{lem}
Let $(K, \Lm_{\text{an}})$ be the structure of subanalytic sets. The induced two-sorted structure $(K, \Gamma_K, \Lm_{\text{an},2})$ is relatively $P$-minimal. 
\end{lem}
\begin{proof}
Recall that this structure has elimination of quantifiers. Let $X \subseteq K \times \Gamma_K^{\ell}$ be a subanalytic set, defined by a formula $\phi(x,\gamma)$. This formula can then be written in the form
\[\phi(x,\gamma):= \phi_1(x) \wedge (\ord f_1(x), \ldots, \ord f_r(x), \gamma) \in P,\]
where $P \subseteq \Gamma_K^{r+ \ell}$ is a Presburger set. 
By $P$-minimality, we may assume that there exists an $\Lring$-formula $\psi_1(x)$, such that $K \models \psi_1(x) \leftrightarrow \phi_1(x)$. Moreover, by the preparation theorem for subanalytic functions, we have that $\ord f_i(x) = \frac1{e_i}\ord a_i(x-c)^{\mu_i}$, and hence there exists a semi-algebraic function $h_i(x)$ such that $\ord f_i(x) = \ord h_i(x)$. It is then clear that the (extended) structure of subanalytic sets is relatively $P$-minimal.
\end{proof}
In general, the notion of relative $P$-minimality may wel be stronger than classical $P$-minimality. {\color{red} Might this be true: both notions are equivalent if $P$-minimality would have the nice properties we want it to have, i.e. cell decompositon and function preparation}